%    Copyright © 2015
%  Eduardo Candido Xavier <eduardo@ic.unicamp.br>
%
%  This work is free. You can redistribute it and/or modify it under the
%  terms of the Do What The Fuck You Want To Public License, Version 2,
%  as published by Sam Hocevar. See the COPYING file for more details.
%
%  DO WHAT THE FUCK YOU WANT TO PUBLIC LICENSE
%                   Version 2, December 2004
%
%   Copyright (C) 2004 Sam Hocevar <sam@hocevar.net>
%
%   Everyone is permitted to copy and distribute verbatim or modified
%   copies of this license document, and changing it is allowed as long
%   as the name is changed.
%
%  DO WHAT THE FUCK YOU WANT TO PUBLIC LICENSE
%   TERMS AND CONDITIONS FOR COPYING, DISTRIBUTION AND MODIFICATION
%
%      0. You just DO WHAT THE FUCK YOU WANT TO.

\documentclass[handout]{beamer}
\usetheme{metropolis}
\beamertemplatetransparentcoveredhigh

\usepackage[portuges]{babel}
\usepackage{graphicx}
\graphicspath{{./figs/}}
\usepackage{listings}
\usepackage{color}
\usepackage{hyperref}
\usepackage{xpatch}
\usepackage[outputdir=build]{minted}

\makeatletter
\AtBeginEnvironment{minted}{\dontdofcolorbox}
\def\dontdofcolorbox{\renewcommand\fcolorbox[4][]{##4}}
\xpatchcmd{\inputminted}{\minted@fvset}{\minted@fvset\dontdofcolorbox}{}{}
\xpatchcmd{\mintinline}{\minted@fvset}{\minted@fvset\dontdofcolorbox}{}{}
\makeatother
\setminted[c]{
  linenos=true,
  breaklines=true,
  encoding=utf8,
  frame=lines,
  framerule=0.5pt,
  autogobble,
  fontsize=\small,
}
\setminted[bash]{
  linenos=true,
  encoding=utf8,
  frame=lines,
  framerule=0.5pt,
  autogobble,
  fontsize=\small
}

\newcommand{\cod}[1]{\mintinline{c}{#1}}


\definecolor{dkgreen}{rgb}{0,0.6,0}
\definecolor{gray}{rgb}{0.5,0.5,0.5}
\definecolor{mauve}{rgb}{0.58,0,0.82}


\definecolor{Purple}{HTML}{911146}
\definecolor{Orange}{HTML}{CF4A30}
\setbeamercolor{alerted text}{fg=Orange}
\setbeamercolor{frametitle}{bg=Purple}
\setbeamercolor{block body}{bg=Purple!20,fg=black}
\setbeamercolor{block title}{bg=Purple!50,fg=black}
\setbeamertemplate{blocks}[rounded][shadow=true]


\newcommand{\setcoverbg}{
  \setbeamertemplate{background}
  {\includegraphics[width=\paperwidth,height=\paperheight]{backgrounds/coverbg}}
}
\newcommand{\setsectionbg}{
  \setbeamertemplate{background}
  {\includegraphics[width=\paperwidth,height=\paperheight]{backgrounds/blank}}
}

\title{Programação Estruturada}
\subtitle{Introdução ao curso}

\author{Professores Emílio Francesquini e Carla Negri Lintzmayer}
\institute{Centro de Matemática, Computação e Cognição\\ Universidade Federal do ABC}
\date{2018.Q3}

\begin{document}

\setcoverbg
\maketitle
\setsectionbg




%%%%%%%%%%%%%%%%%%%%%%%%%%%%%%%%%%%%
%%%%%%%%%%%%%%%%%%%%%%%%%%%%%%%%%%%%
%%%%%%%%%%%%%%%%%%%%%%%%%%%%%%%%%%%%
%%%%%%%%%%%%%%%%%%%%%%%%%%%%%%%%%%%%
%%%%%%%%%%%%%%%%%%%%%%%%%%%%%%%%%%%%
%%%%%%%%%%%%%%%%%%%%%%%%%%%%%%%%%%%%
\section{O que vamos aprender neste curso}

%%%%%%%%%%%%%%%%%%%%%%%%%%%%%%%%%%%%
\begin{frame}{O que vamos aprender neste curso}

    \begin{itemize}[<+->]
        \item Como usar um computador para resolver problemas
        \begin{enumerate}
            \item Definiremos um problema a ser revolvido
            \item Discutiremos uma solução para o problema
            \item Descreveremos um \textbf{algoritmo} para resolver o problema
            \begin{itemize}
                \item Sequência \textbf{bem definida} de comandos e passos
            \end{itemize}
            \item \textbf{Implementaremos} este algoritmo e criaremos um \textbf{programa}
            \begin{itemize}
                \item Sequência de comandos e passos que um computador deve executar
            \end{itemize}
        \end{enumerate}
    \end{itemize}
\end{frame}

%%%%%%%%%%%%%%%%%%%%%%%%%%%%%%%%%%%%
\begin{frame}[fragile]{O que vamos aprender neste curso}

    \textbf{Problema:} Dado um número $x$, quais são os divisores de $x$?

    \pause
    \textbf{Discussão:} Os candidatos a serem divisores de $x$ só podem estar no conjunto $\{1,2,3,\ldots,x\}$.
    Podemos testar cada um desses números e verificar se eles dividem $x$.

    \pause
    \textbf{Algoritmo:}
    \begin{enumerate}
        \item Para todos os números inteiros $r \in \{1,2,\ldots,x\}$:
        \begin{enumerate}
            \item Se o resto da divisão de $x$ por $r$ é zero, então $r$ é divisor de $x$
        \end{enumerate}
    \end{enumerate}
\end{frame}

%%%%%%%%%%%%%%%%%%%%%%%%%%%%%%%%%%%%
\begin{frame}[fragile]{O que vamos aprender neste curso}

    \textbf{Problema:} Dado um número $x$, quais são os divisores de $x$?

    \textbf{Implementação:}
    \begin{minted}{c}
        #include <stdio.h>

        int main() {
            int x, i;
            scanf("%d", &x); // lendo um numero do teclado
            for (i = 1; i <= x; i++) {
                if (x % i == 0)
                    printf("%d\n", i); // imprimindo um divisor
            }
            return 0;
        }
    \end{minted}
\end{frame}

%%%%%%%%%%%%%%%%%%%%%%%%%%%%%%%%%%%%
\begin{frame}{O que vamos aprender neste curso}

    \begin{itemize}[<+->]
        \item Um algoritmo pode ser descrito de várias formas, como em português ou em uma \textbf{linguagem de programação}
        \begin{itemize}
            \item Conjunto de instruções e regras para gerar um programa de computador
            \item Usadas para criar arquivos de texto comum: códigos-fonte
        \end{itemize}
        \item Aprenderemos a {\bf implementar} um algoritmo em linguagem \textbf{C}
        \item A vantagem de se implementar um algoritmo em uma linguagem de programação é que podemos, a partir daí, criar um \textbf{programa} que usa o computador para resolver o problema
    \end{itemize}
\end{frame}

%%%%%%%%%%%%%%%%%%%%%%%%%%%%%%%%%%%%
\begin{frame}{Por que aprender algoritmos e programação?}

    \begin{itemize}[<+->]
        \item Atividade básica de qualquer bom computeiro
        \item Para ser capaz de automatizar algum processo
        \item Para criar ferramentas/protótipos, você deverá fazer simulações para a realização de testes preliminares
        \item Para enxergar situações onde uma solução computacional pode trazer benefícios
        \item Para testar hipóteses
        \item Para resolver sistemas complexos de equações que não necessariamente podem ser resolvidos por softwares padrões (como MatLab)
        \item Posso ter algum retorno financeiro com isso!
        \item Porque é legal e desfiador!
    \end{itemize}
\end{frame}

%%%%%%%%%%%%%%%%%%%%%%%%%%%%%%%%%%%%
\begin{frame}{O que será necessário}

    \begin{itemize}[<+->]
        \item Papel e caneta em todas as aulas
        \item Acesso a um computador
        \item Um {\bf editor de texto} simples (vim, emacs, gedit, kyle, etc.)
        \item Um {\bf compilador}, que transforma o código escrito em C em um programa executável (gcc)
        \item Capacidade para resolver problemas técnicos: como todo bom desenvolvedor, será muito importante você encontrar soluções para problemas técnicos
        \begin{itemize}
            \item Aprendam a usar o Google
        \end{itemize}
    \end{itemize}
\end{frame}

%%%%%%%%%%%%%%%%%%%%%%%%%%%%%%%%%%%%
\begin{frame}{O que será necessário}

    Para ir bem neste curso:
    \begin{itemize}[<+->]
        \item Faça todos os laboratórios
        \item Faça e implemente as listas de exercícios
        \item E, finalmente, faça e implemente as listas de exercícios
    \end{itemize}
\end{frame}

\end{document}
