%    Copyright © 2015
%  Eduardo Candido Xavier <eduardo@ic.unicamp.br>
%
%  This work is free. You can redistribute it and/or modify it under the
%  terms of the Do What The Fuck You Want To Public License, Version 2,
%  as published by Sam Hocevar. See the COPYING file for more details.
%
%  DO WHAT THE FUCK YOU WANT TO PUBLIC LICENSE
%                   Version 2, December 2004
%
%   Copyright (C) 2004 Sam Hocevar <sam@hocevar.net>
%
%   Everyone is permitted to copy and distribute verbatim or modified
%   copies of this license document, and changing it is allowed as long
%   as the name is changed.
%
%  DO WHAT THE FUCK YOU WANT TO PUBLIC LICENSE
%   TERMS AND CONDITIONS FOR COPYING, DISTRIBUTION AND MODIFICATION
%
%      0. You just DO WHAT THE FUCK YOU WANT TO.

\documentclass[handout]{beamer}
\usetheme{metropolis}
\beamertemplatetransparentcoveredhigh

\usepackage[portuges]{babel}
\usepackage{graphicx}
\graphicspath{{./figs/}}
\usepackage{listings}
\usepackage{color}
\usepackage{hyperref}
\usepackage{xpatch}
\usepackage[outputdir=build]{minted}

\makeatletter
\AtBeginEnvironment{minted}{\dontdofcolorbox}
\def\dontdofcolorbox{\renewcommand\fcolorbox[4][]{##4}}
\xpatchcmd{\inputminted}{\minted@fvset}{\minted@fvset\dontdofcolorbox}{}{}
\xpatchcmd{\mintinline}{\minted@fvset}{\minted@fvset\dontdofcolorbox}{}{}
\makeatother
\setminted[c]{
  linenos=true,
  breaklines=true,
  encoding=utf8,
  frame=lines,
  framerule=0.5pt,
  autogobble,
  fontsize=\small,
}
\setminted[bash]{
  linenos=true,
  encoding=utf8,
  frame=lines,
  framerule=0.5pt,
  autogobble,
  fontsize=\small
}

\newcommand{\cod}[1]{\mintinline{c}{#1}}


\definecolor{dkgreen}{rgb}{0,0.6,0}
\definecolor{gray}{rgb}{0.5,0.5,0.5}
\definecolor{mauve}{rgb}{0.58,0,0.82}


\definecolor{Purple}{HTML}{911146}
\definecolor{Orange}{HTML}{CF4A30}
\setbeamercolor{alerted text}{fg=Orange}
\setbeamercolor{frametitle}{bg=Purple}
\setbeamercolor{block body}{bg=Purple!20,fg=black}
\setbeamercolor{block title}{bg=Purple!50,fg=black}
\setbeamertemplate{blocks}[rounded][shadow=true]


\newcommand{\setcoverbg}{
  \setbeamertemplate{background}
  {\includegraphics[width=\paperwidth,height=\paperheight]{backgrounds/coverbg}}
}
\newcommand{\setsectionbg}{
  \setbeamertemplate{background}
  {\includegraphics[width=\paperwidth,height=\paperheight]{backgrounds/blank}}
}

\title{Programação Estruturada}
\subtitle{Operações e expressões relacionais e lógicas}

\author{Professores Emílio Francesquini e Carla Negri Lintzmayer}
\institute{Centro de Matemática, Computação e Cognição\\ Universidade Federal do ABC}
\date{2018.Q3}

\begin{document}

\setcoverbg
\maketitle
\setsectionbg




%%%%%%%%%%%%%%%%%%%%%%%%%%%%%%%%%%%%%%%%%%%%%%%%
\begin{frame}[fragile]{Expressão}
    \begin{itemize}[<+->]
        \item Já vimos que constantes e variáveis são expressões.
        \begin{minted}{c}
            a = 10;
            a = b;
        \end{minted}

        \item Vimos também que operações aritméticas também são expressões.
        \begin{minted}{c}
            a = 2 + 2;
            a = 10 / 3;
            a = a + 1;
            a = (2 + 5) * (10 % 4);
        \end{minted}
  \end{itemize}
\end{frame}

%%%%%%%%%%%%%%%%%%%%%%%%%%%%%%%%%%%%%%%%%%%%%%%%
%%%%%%%%%%%%%%%%%%%%%%%%%%%%%%%%%%%%%%%%%%%%%%%%
%%%%%%%%%%%%%%%%%%%%%%%%%%%%%%%%%%%%%%%%%%%%%%%%
%%%%%%%%%%%%%%%%%%%%%%%%%%%%%%%%%%%%%%%%%%%%%%%%
%%%%%%%%%%%%%%%%%%%%%%%%%%%%%%%%%%%%%%%%%%%%%%%%
%%%%%%%%%%%%%%%%%%%%%%%%%%%%%%%%%%%%%%%%%%%%%%%%
%%%%%%%%%%%%%%%%%%%%%%%%%%%%%%%%%%%%%%%%%%%%%%%%

\section{Expressões relacionais}

%%%%%%%%%%%%%%%%%%%%%%%%%%%%%%%%%%%%%%%%%%%%%%%%
\begin{frame}[fragile]{Expressões relacionais}

    \textbf{Expressões relacionais} são aquelas que realizam uma comparação entre duas expressões e têm valor

    \begin{itemize}
        \item {\bf zero ($0$)}, se o resultado é falso
        \item {\bf um ($1$) ou qualquer outro número diferente de zero}, se o resultado é verdadeiro
    \end{itemize}
\end{frame}

%%%%%%%%%%%%%%%%%%%%%%%%%%%%%%%%%%%%%%%%%%%%%%%%
\begin{frame}[fragile]{Operadores relacionais}

    Os operadores relacionais da linguagem C são:

    \begin{center}
        \begin{tabular}{|c|l|}
            \hline
            \cod{==} & igualdade \\
            \cod{!=} & diferença \\
            \cod{>}  & maior que \\
            \cod{<}  & menor que \\
            \cod{>=} & maior ou igual a \\
            \cod{<=} & menor ou igual a \\
            \hline
        \end{tabular}
    \end{center}
\end{frame}

%%%%%%%%%%%%%%%%%%%%%%%%%%%%%%%%%%%%%%%%%%%%%%%%
\begin{frame}[fragile]{Expressões relacionais}

    \begin{itemize}[<+->]
        \item {\it expressão} \cod{==} {\it expressão} \\
        Tem valor verdadeiro quando as expressões forem iguais.
        \begin{itemize}
            \item \cod{9 == 9}: tem resultado 1, verdadeiro
            \item \cod{9 == 10}: tem resultado 0, falso
        \end{itemize}

        \item {\it expressão} \cod{!=} {\it expressão} \\
        Tem valor verdadeiro quando as expressões forem diferentes.
        \begin{itemize}
            \item \cod{9 != 9}: tem resultado 0, falso
            \item \cod{9 != 10}: tem resultado 1, verdadeiro
        \end{itemize}
    \end{itemize}

\end{frame}

%%%%%%%%%%%%%%%%%%%%%%%%%%%%%%%%%%%%%%%%%%%%%%%%
\begin{frame}[fragile]{Expressões relacionais}

    \begin{itemize}
        \item {\it expressão} \cod{>} {\it expressão} \\
        Tem valor verdadeiro quando a expressão da esquerda tiver valor maior do que a expressão da direita.
        \begin{itemize}
            \item \cod{9 > 5}: tem resultado 1, verdadeiro
            \item \cod{9 > 9}: tem resultado 0, falso
        \end{itemize}

        \item {\it expressão} \cod{<} {\it expressão} \\
        Tem valor verdadeiro quando a expressão da esquerda tiver valor menor do que a expressão da direita.
        \begin{itemize}
            \item \cod{9 < 5}: tem resultado 0, falso
            \item \cod{-5 < 0}: tem resultado 1, verdadeiro
        \end{itemize}
    \end{itemize}

\end{frame}

%%%%%%%%%%%%%%%%%%%%%%%%%%%%%%%%%%%%%%%%%%%%%%%%
\begin{frame}[fragile]{Expressões relacionais}

    \begin{itemize}
        \item {\it expressão} \cod{>=} {\it expressão} \\
        Tem valor verdadeiro quando a expressão da esquerda tiver valor maior ou igual ao valor da expressão da direita.
        \begin{itemize}
            \item \cod{9 >= 5}: tem resultado 1, verdadeiro
            \item \cod{-5 >= 0}: tem resultado 0, falso
        \end{itemize}

        \item {\it expressão} \cod{<=} {\it expressão} \\
        Retorna verdadeiro quando a expressão da esquerda tiver valor menor ou igual ao valor da expressão da direita.
        \begin{itemize}
            \item \cod{9 <= 5}: tem resultado 0, falso
            \item \cod{9 <= 9}: tem resultado 1, verdadeiro
        \end{itemize}
    \end{itemize}

\end{frame}

%%%%%%%%%%%%%%%%%%%%%%%%%%%%%%%%%%%%%%%%%%%%%%%%
\begin{frame}[fragile]{Exemplo}

    O que será impresso pelo programa?

    \begin{minted}{c}
        #include <stdio.h>

        int main() {
            int a, b;

            printf("%d\n", 9 > 3);
            printf("%d\n", (3*4)/2 != (2*3));

            a = 1;
            b = -1;
            printf("%d\n", a != b);

            return 0;
        }
    \end{minted}
\end{frame}

%%%%%%%%%%%%%%%%%%%%%%%%%%%%%%%%%%%%%%%%%%%%%%%%%
%%%%%%%%%%%%%%%%%%%%%%%%%%%%%%%%%%%%%%%%%%%%%%%%%
%%%%%%%%%%%%%%%%%%%%%%%%%%%%%%%%%%%%%%%%%%%%%%%%%
%%%%%%%%%%%%%%%%%%%%%%%%%%%%%%%%%%%%%%%%%%%%%%%%%
%%%%%%%%%%%%%%%%%%%%%%%%%%%%%%%%%%%%%%%%%%%%%%%%%
%%%%%%%%%%%%%%%%%%%%%%%%%%%%%%%%%%%%%%%%%%%%%%%%%

\section{Expressões lógicas}

%%%%%%%%%%%%%%%%%%%%%%%%%%%%%%%%%%%%%%%%%%%%%%%%
\begin{frame}[fragile]{Expressões lógicas}

    \textbf{Expressões lógicas} são aquelas que realizam uma operação lógica (conjunção, disjunção, negação, etc.) e têm valor verdadeiro ou falso (como as expressões relacionais).

\end{frame}

%%%%%%%%%%%%%%%%%%%%%%%%%%%%%%%%%%%%%%%%%%%%%%%%
\begin{frame}[fragile]{Operadores lógicos}
    Na linguagem C temos os seguintes operadores lógicos:

    \begin{center}
        \begin{tabular}{|c|l|}
            \hline
            \cod{&&} & operador de conjunção (E) \\
            \cod{||} & operador de disjunção (OU) \\
            \cod{!} & operador de negação (NÃO) \\
            \hline
        \end{tabular}
    \end{center}

\end{frame}

%%%%%%%%%%%%%%%%%%%%%%%%%%%%%%%%%%%%%%%%%%%%%%%%
\begin{frame}[fragile]{Expressões lógicas}

    \begin{itemize}
        \item {\it expressão} \cod{&&} {\it expressão} \\
        Retorna verdadeiro quando ambas as expressões são verdadeiras.

        \begin{center}
            \begin{tabular}{|c|c||c|}
                \hline
                $Op_1$ & $Op_2$ & $Valor$ \\ \hline
                V & V & V \\
                V & F & F \\
                F & V & F \\
                F & F & F \\
                \hline
            \end{tabular}
        \end{center}
    \end{itemize}

    Qual o valor de \cod{v} abaixo?
    \begin{minted}{c}
        a = 0;
        b = 0;
        v = (a == 0 && b == 0);
    \end{minted}
\end{frame}

%%%%%%%%%%%%%%%%%%%%%%%%%%%%%%%%%%%%%%%%%%%%%%%%
\begin{frame}[fragile]{Expressões lógicas}

    \begin{itemize}
        \item {\it expressão} \cod{||} {\it expressão} \\
        Retorna verdadeiro quando pelo menos uma das expressões é verdadeira.

        \begin{center}
            \begin{tabular}{|c|c||c|}
                \hline
                $Op_1$ & $Op_2$ & $Valor$ \\ \hline
                V & V & V \\
                V & F & V \\
                F & V & V \\
                F & F & F \\
                \hline
            \end{tabular}
        \end{center}
    \end{itemize}

    Qual o valor de \cod{v} abaixo?
    \begin{minted}{c}
        a = 0;
        b = 1;
        v = (a == 0 || b == 0);
    \end{minted}
\end{frame}

%%%%%%%%%%%%%%%%%%%%%%%%%%%%%%%%%%%%%%%%%%%%%%%%
\begin{frame}[fragile]{Expressões lógicas}

    \begin{itemize}
        \item \cod{!} {\it expressão} \\
        Retorna verdadeiro quando a expressão é falsa e vice-versa.

        \begin{center}
            \begin{tabular}{|c||c|}
                \hline
                $Op_1$ & $Valor$ \\ \hline
                V & F \\
                F & V \\
                \hline
            \end{tabular}
        \end{center}
    \end{itemize}

    Qual o valor de \cod{v} abaixo?
    \begin{minted}{c}
        a = 0;
        b = 1;
        v = !(a != b);
    \end{minted}
\end{frame}

%%%%%%%%%%%%%%%%%%%%%%%%%%%%%%%%%%%%%%%%%%%%%%%%
\begin{frame}[fragile]{Exemplo}
    O que será impresso pelo programa?

    \begin{minted}{c}
        #include <stdio.h>

        int main() {
            printf("%d\n", (8 > 9) && (10 != 2));
            printf("%d\n", (14 > 100) || (2 > 1));
            printf("%d\n", (!(14 > 100) && !(1 > 2)));
            return 0;
        }
    \end{minted}
\end{frame}

\end{document}
