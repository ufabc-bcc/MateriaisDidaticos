% Copyright © 2015
% Eduardo Candido Xavier <eduardo@ic.unicamp.br>
%
% This work is free. You can redistribute it and/or modify it under the
% terms of the Do What The Fuck You Want To Public License, Version 2,
% as published by Sam Hocevar. See the COPYING file for more details.
%
% DO WHAT THE FUCK YOU WANT TO PUBLIC LICENSE
% Version 2, December 2004
%
% Copyright (C) 2004 Sam Hocevar <sam@hocevar.net>
%
% Everyone is permitted to copy and distribute verbatim or modified
% copies of this license document, and changing it is allowed as long
% as the name is changed.
%
% DO WHAT THE FUCK YOU WANT TO PUBLIC LICENSE
% TERMS AND CONDITIONS FOR COPYING, DISTRIBUTION AND MODIFICATION
%
% 0. You just DO WHAT THE FUCK YOU WANT TO.

\documentclass[handout]{beamer}
\usetheme{metropolis}
\beamertemplatetransparentcoveredhigh

\usepackage[portuges]{babel}
\usepackage{graphicx}
\graphicspath{{./figs/}}
\usepackage{listings}
\usepackage{color}
\usepackage{hyperref}
\usepackage{xpatch}
\usepackage[outputdir=build]{minted}

\makeatletter
\AtBeginEnvironment{minted}{\dontdofcolorbox}
\def\dontdofcolorbox{\renewcommand\fcolorbox[4][]{##4}}
\xpatchcmd{\inputminted}{\minted@fvset}{\minted@fvset\dontdofcolorbox}{}{}
\xpatchcmd{\mintinline}{\minted@fvset}{\minted@fvset\dontdofcolorbox}{}{}
\makeatother
\setminted[c]{
  linenos=true,
  breaklines=true,
  encoding=utf8,
  frame=lines,
  framerule=0.5pt,
  autogobble,
  fontsize=\small,
}
\setminted[bash]{
  linenos=true,
  encoding=utf8,
  frame=lines,
  framerule=0.5pt,
  autogobble,
  fontsize=\small
}

\newcommand{\cod}[1]{\mintinline{c}{#1}}


\definecolor{dkgreen}{rgb}{0,0.6,0}
\definecolor{gray}{rgb}{0.5,0.5,0.5}
\definecolor{mauve}{rgb}{0.58,0,0.82}


\definecolor{Purple}{HTML}{911146}
\definecolor{Orange}{HTML}{CF4A30}
\setbeamercolor{alerted text}{fg=Orange}
\setbeamercolor{frametitle}{bg=Purple}
\setbeamercolor{block body}{bg=Purple!20,fg=black}
\setbeamercolor{block title}{bg=Purple!50,fg=black}
\setbeamertemplate{blocks}[rounded][shadow=true]


\newcommand{\setcoverbg}{
  \setbeamertemplate{background}
  {\includegraphics[width=\paperwidth,height=\paperheight]{backgrounds/coverbg}}
}
\newcommand{\setsectionbg}{
  \setbeamertemplate{background}
  {\includegraphics[width=\paperwidth,height=\paperheight]{backgrounds/blank}}
}

\title{Programação Estruturada}
\subtitle{Funções}

\author{Professores Emílio Francesquini e Carla Negri Lintzmayer}
\institute{Centro de Matemática, Computação e Cognição\\ Universidade Federal do ABC}
\date{2018.Q3}

\begin{document}

\setcoverbg
\maketitle
\setsectionbg

%%%%%%%%%%%%%%%%%%%%%%%%%%%%%%%%%%%%%%%
%%%%%%%%%%%%%%%%%%%%%%%%%%%%%%%%%%%%%%%
%%%%%%%%%%%%%%%%%%%%%%%%%%%%%%%%%%%%%%%
%%%%%%%%%%%%%%%%%%%%%%%%%%%%%%%%%%%%%%%
%%%%%%%%%%%%%%%%%%%%%%%%%%%%%%%%%%%%%%%
%%%%%%%%%%%%%%%%%%%%%%%%%%%%%%%%%%%%%%%

\section{Funções}

%%%%%%%%%%%%%%%%%%%%%%%%%%%%%%%%%%%%%%%
\begin{frame}[fragile]{Funções} 
    
    \begin{itemize}[<+->]
        \item Um ponto chave na resolução de um problema complexo é conseguir ``quebrá-lo'' em subproblemas menores.
        \item Ao criarmos um programa para resolver um problema, é crítico quebrar um código grande em partes menores, fáceis de serem entendidas e administradas. 
        \item Isto é conhecido como modularizacão, e é empregado em qualquer projeto de engenharia envolvendo a construção de um sistema complexo.
    \end{itemize}
\end{frame}

%%%%%%%%%%%%%%%%%%%%%%%%%%%%%%%%%%%%%%%
\begin{frame}[fragile]{Funções} 

    \begin{block}{Funções}
        São estruturas que agrupam um conjunto de comandos, que são executados quando a função é chamada/invocada.
    \end{block}

    \begin{itemize}
        \item Vocês já usaram algumas funções como {\bf scanf} e {\bf printf}.
        \item Algumas funções podem devolver algum valor ao final de sua execução:

        \begin{minted}{c}
            x = sqrt(4);
        \end{minted}

        \item Vamos aprender como criar e usar funções em C.
    \end{itemize}
\end{frame}

%%%%%%%%%%%%%%%%%%%%%%%%%%%%%%%%%%%%%%%
\begin{frame}{Por que utilizar funções?}

    \begin{itemize}[<+->]
        \item Evitar que os blocos do programa fiquem grandes demais e, por consequência, mais difíceis de ler e entender.
        \item Separar o programa em partes que possam ser logicamente compreendidas de forma isolada.
        \item Permitir o reaproveitamento de código já construído (por você ou por outros programadores).
        \item Evitar que um trecho de código seja repetido várias vezes dentro de um mesmo programa, minimizando erros e facilitando alterações. 
    \end{itemize}
\end{frame}

%%%%%%%%%%%%%%%%%%%%%%%%%%%%%%%%%%%%%%%
%%%%%%%%%%%%%%%%%%%%%%%%%%%%%%%%%%%%%%%
%%%%%%%%%%%%%%%%%%%%%%%%%%%%%%%%%%%%%%%

\subsection{Definindo uma função}

%%%%%%%%%%%%%%%%%%%%%%%%%%%%%%%%%%%%%%%
\begin{frame}[fragile]{Definindo uma função}

    Uma função é definida da seguinte forma:
    
    \begin{minted}{c}
        tipo_retorno nome(tipo parâmetro1,..., tipo parâmetroN) {
            comandos;
            return valor_de_retorno;
        }
    \end{minted}

    \begin{itemize}
        \item Toda função deve ter um tipo ({\bf int, char, float, void, etc}). Esse tipo determina qual será o tipo de seu valor de retorno.
        \item Os {\bf parâmetros} são variáveis, que são inicializadas com valores indicados durante a invocação da função.
        \item O comando {\bf return} devolve para o invocador da função o resultado da execução desta.
    \end{itemize}
\end{frame}

%%%%%%%%%%%%%%%%%%%%%%%%%%%%%%%%%%%%%%%
\begin{frame}[fragile]{Definindo uma função: exemplo}

    A função abaixo recebe como parâmetro dois valores inteiros. 
    Ela faz a soma destes valores e devolve o resultado.

    \begin{minted}{c}
        int soma(int a, int b) {
            int c;
            c = a + b;
            return c;
        }
    \end{minted}

    \begin{itemize}
        \item Note que o valor de retorno (variável {\bf c}) é do mesmo tipo da função.
        \item Quando o comando {\bf return} é executado, a função para de executar e retorna o valor indicado para quem fez a invocação (ou chamada) da função.
    \end{itemize}
\end{frame}

%%%%%%%%%%%%%%%%%%%%%%%%%%%%%%%%%%%%%%%
\begin{frame}[fragile]{Definindo uma função: exemplo}

    \begin{minted}[fontsize=\scriptsize]{c}
        int soma(int a, int b) {
            int c;
            c = a + b;
            return c;
        }
    \end{minted}

    \small
    Qualquer função pode invocar esta função \cod{soma}, passando como parâmetro dois valores inteiros, que serão atribuídos para as variáveis {\bf a} e {\bf b}, respectivamente.

    \begin{minted}[fontsize=\scriptsize]{c}
        int main() {
            int r;
            r = soma(12, 90);
            printf("r = %d\n", r);
            r = soma (-9, 45);
            printf("r = %d\n", r);
            return 0;
        }
    \end{minted}
\end{frame}

%%%%%%%%%%%%%%%%%%%%%%%%%%%%%%%%%%%%%%%
\begin{frame}[fragile]{Definindo uma função: exemplo}

    \begin{minted}[fontsize=\scriptsize]{c}
        #include <stdio.h>

        int soma(int a, int b) {
            int c;
            c = a + b;
            return c;
        }

        int main() {
            int res, x1 = 4, x2 = -10;
            res = soma(5, 6);
            printf("Primeira soma: %d\n", res);
            res = soma(x1, x2);
            printf("Segunda soma: %d\n",res);
            return 0;
        }
    \end{minted}
\end{frame}

%%%%%%%%%%%%%%%%%%%%%%%%%%%%%%%%%%%%%%%
\begin{frame}[fragile]{Execução de um programa com funções}
    \begin{itemize}[<+->]
        \item Qualquer programa começa executando os comandos da função {\bf main}.
        \item Quando se encontra a chamada para uma função, o fluxo de execução passa para ela e são executados os comandos até que um {\bf return} seja encontrado ou o fim da função seja alcançado.
        \item Depois disso, o fluxo de execução volta para o ponto onde a chamada da função ocorreu.
    \end{itemize}
\end{frame}

%%%%%%%%%%%%%%%%%%%%%%%%%%%%%%%%%%%%%%%
\begin{frame}[fragile]{Definindo uma função: exemplo}

    A lista de parâmetros de uma função pode ser vazia.

    \begin{minted}[fontsize=\scriptsize]{c}
        int leInteiro() {
            int c;
            printf("Digite um número: ");
            scanf("%d", &c);
            return c;
        }
    \end{minted}

    O retorno será usado pelo invocador da função:

    \begin{minted}[fontsize=\scriptsize]{c}
        int main() {
            int r;
            r = leInteiro();
            printf("Numero digitado: %d\n", r);
            return 0;
        }
    \end{minted}
\end{frame}

%%%%%%%%%%%%%%%%%%%%%%%%%%%%%%%%%%%%%%%
\begin{frame}[fragile]{Definindo uma função: exemplo}

    \begin{minted}{c}
        #include <stdio.h>

        int leInteiro() {
            int c;
            printf("Digite um numero: ");
            scanf("%d", &c);
            return c;
        }

        int main() {
            int r;
            r = leInteiro();
            printf("Numero digitado: %d\n", r);
            return 0;
        }
    \end{minted}
\end{frame}

%%%%%%%%%%%%%%%%%%%%%%%%%%%%%%%%%%%%%%%
\begin{frame}[fragile]{Exemplo de função}

    \small
    A expressão contida dentro do comando {\bf return} é chamada de \textit{valor de retorno} (é a resposta da função).
    Nada após esse comando será executado.

    \begin{minted}[fontsize=\tiny]{c}
        #include <stdio.h>

        int leInteiro() {
            int c;
            printf("Digite um numero: ");
            scanf("%d", &c);
            return c;
            printf("Bla bla bla!\n");
        }

        int soma (int a, int b) {
            return a + b;
        }
        
        int main() {
            int x1, x2, res;
            x1 = leInteiro();
            x2 = leInteiro();
            res = soma(x1, x2);
            printf("Soma eh: %d\n", res);
            return 0;
        }
    \end{minted}
\end{frame}

%%%%%%%%%%%%%%%%%%%%%%%%%%%%%%%%%%%%%%%
%%%%%%%%%%%%%%%%%%%%%%%%%%%%%%%%%%%%%%%
%%%%%%%%%%%%%%%%%%%%%%%%%%%%%%%%%%%%%%%

\subsection{Invocando uma função}

%%%%%%%%%%%%%%%%%%%%%%%%%%%%%%%%%%%%%%%
\begin{frame}[fragile]{Invocando uma função}

    \begin{itemize}
        \item Uma forma clássica de realizarmos a invocação (ou chamada) de uma função é atribuindo o seu valor a uma variável:
        \begin{minted}{c}
            x = soma(4, 2);
        \end{minted}

        \pause

        \item Na verdade, o resultado da chamada de uma função é uma expressão e pode ser usado em qualquer lugar que aceite uma expressão:
        \begin{minted}{c}
            printf("Soma de a e b: %d\n", soma(a, b));
        \end{minted}
    \end{itemize}
\end{frame}

%%%%%%%%%%%%%%%%%%%%%%%%%%%%%%%%%%%%%%%
\begin{frame}[fragile]{Invocando uma função}

    Na chamada da função, para cada um dos parâmetros desta devemos fornecer um valor de mesmo tipo e na mesma ordem da declaração.
    \begin{minted}{c}
        #include <stdio.h>

        int somaComMensagem(int a, int b, char texto[100]) {
            int c = a + b;
            printf("%s %d\n", texto, c);
            return c;
        }

        int main() {
            somaComMensagem(4, 5, "Resultado da soma:");
            return 0;
        }
    \end{minted}
\end{frame}

%%%%%%%%%%%%%%%%%%%%%%%%%%%%%%%%%%%%%%%
\begin{frame}[fragile]{Invocando uma função}

    A saída do programa anterior será:

    \begin{minted}{bash}
        Resultado da soma: 9
    \end{minted}

    \pause

    Já a chamada abaixo gerará um erro de compilação:
    \begin{minted}{c}
        int main() {
            somaComMensagem(4, "Resultado da soma:",  5);
            return 0;
        }
    \end{minted}
\end{frame}

%%%%%%%%%%%%%%%%%%%%%%%%%%%%%%%%%%%%%%%
\begin{frame}[fragile]{Invocando uma função}

    \small
    \begin{itemize}
        \item Ao chamar uma função passando variáveis {\bf simples} como parâmetros, estamos usando apenas os seus valores, que serão copiados para as variáveis parâmetros da função.
        \item Os valores das variáveis na chamada da função não são afetados por alterações dentro da função.

        \begin{minted}[fontsize=\tiny]{c}
            #include <stdio.h>

            int incr(int x) {
                x = x + 1;
                return x;
            }

            int main() {
                int a = 2, b;
                b = incr(a);
                printf("a = %d, b = %d\n", a, b);
                return 0;
            }
        \end{minted}

        \item O que será impresso? O valor de {\bf a} é alterado pela função {\bf incr}?
    \end{itemize}
\end{frame}

%%%%%%%%%%%%%%%%%%%%%%%%%%%%%%%%%%%%%%%
%%%%%%%%%%%%%%%%%%%%%%%%%%%%%%%%%%%%%%%
%%%%%%%%%%%%%%%%%%%%%%%%%%%%%%%%%%%%%%%
%%%%%%%%%%%%%%%%%%%%%%%%%%%%%%%%%%%%%%%
%%%%%%%%%%%%%%%%%%%%%%%%%%%%%%%%%%%%%%%
%%%%%%%%%%%%%%%%%%%%%%%%%%%%%%%%%%%%%%%

\section{O tipo void}

%%%%%%%%%%%%%%%%%%%%%%%%%%%%%%%%%%%%%%%
\begin{frame}{O tipo {\bf void}}

    \begin{itemize}
        \item O ``tipo'' {\bf void} é um tipo especial.
        \item Ele representa  ``nada'', ou seja, uma variável desse tipo armazena conteúdo indeterminado, e uma função desse tipo retorna um conteúdo indeterminado.
        \item Em geral, este tipo é utilizado para indicar que uma função não retorna nenhum valor.
    \end{itemize}  
\end{frame}

%%%%%%%%%%%%%%%%%%%%%%%%%%%%%%%%%%%%%%%
\begin{frame}[fragile]{O tipo {\bf void}}

    \begin{itemize}
        \item Por exemplo, a função abaixo imprime o número que for passado para ela como parâmetro e não devolve nada.
        \item Neste caso não utilizamos o comando {\bf return}.
        
        \begin{minted}{c}
            void imprimeInteiro(int numero) {
                printf("Número %d\n", numero);
            }
        \end{minted}

    \end{itemize}
\end{frame}

%%%%%%%%%%%%%%%%%%%%%%%%%%%%%%%%%%%%%%%
\begin{frame}[fragile]{O tipo {\bf void}}

    \begin{minted}{c}
        #include <stdio.h>

        void imprimeInteiro(int numero) {
            printf("Número %d\n", numero);
        }

        int main() {
            imprimeInteiro(10);
            imprimeInteiro(20);
            return 0;
        }
    \end{minted}
\end{frame}

%%%%%%%%%%%%%%%%%%%%%%%%%%%%%%%%%%%%%%%
%%%%%%%%%%%%%%%%%%%%%%%%%%%%%%%%%%%%%%%
%%%%%%%%%%%%%%%%%%%%%%%%%%%%%%%%%%%%%%%
%%%%%%%%%%%%%%%%%%%%%%%%%%%%%%%%%%%%%%%
%%%%%%%%%%%%%%%%%%%%%%%%%%%%%%%%%%%%%%%
%%%%%%%%%%%%%%%%%%%%%%%%%%%%%%%%%%%%%%%

\section {A função {\bf main}}

%%%%%%%%%%%%%%%%%%%%%%%%%%%%%%%%%%%%%%%
\begin{frame}[fragile]{A função {\bf main}}

    \begin{itemize}
        \item O programa principal é uma função especial, que possui um tipo fixo ({\bf int}) e é invocada automaticamente pelo sistema operacional quando este inicia a execução do programa.
        \item Quando utilizado, o comando {\bf return} informa ao sistema operacional se o programa funcionou corretamente ou não. O padrão é que um programa retorne zero caso tenha funcionado corretamente ou qualquer outro valor caso contrário.
        
        \begin{minted}{c}
            #include <stdio.h>

            int main() {
                printf("Ola turma!\n");
                return 0;
            }
        \end{minted}
    \end{itemize}
\end{frame}

%%%%%%%%%%%%%%%%%%%%%%%%%%%%%%%%%%%%%%%
%%%%%%%%%%%%%%%%%%%%%%%%%%%%%%%%%%%%%%%
%%%%%%%%%%%%%%%%%%%%%%%%%%%%%%%%%%%%%%%
%%%%%%%%%%%%%%%%%%%%%%%%%%%%%%%%%%%%%%%
%%%%%%%%%%%%%%%%%%%%%%%%%%%%%%%%%%%%%%%
%%%%%%%%%%%%%%%%%%%%%%%%%%%%%%%%%%%%%%%

\section{Protótipo de funções}

%%%%%%%%%%%%%%%%%%%%%%%%%%%%%%%%%%%%%%%
\begin{frame}[fragile]{Protótipo de funções: definindo funções depois da {\bf main}}

    Até o momento, aprendemos que devemos definir as funções antes do programa principal.
    O que ocorreria se declarássemos depois?

    \begin{minted}{c}
        #include <stdio.h>

        int main() {
            float a = 0, b = 5;
            printf("%f\n", soma(a, b));
            return 0;
        }

        float soma(float op1, float op2) {
            return op1 + op2;
        }
    \end{minted}

    \pause
    Dependendo do compilador, ocorre um erro de compilação!
\end{frame}

%%%%%%%%%%%%%%%%%%%%%%%%%%%%%%%%%%%%%%%
\begin{frame}[fragile]{Protótipo de funções: declarando uma função sem defini-la}

    \begin{itemize}[<+->]
        \item Para organizar melhor um programa e podermos implementar funções em partes distintas do arquivo fonte, utilizamos {\bf protótipos de funções}.
        \item Protótipos de funções correspondem a primeira linha da definição de uma função, contendo: tipo de retorno, nome da função, parâmetros e por fim {\bf um ponto e vírgula}.

        \begin{minted}{c}
            tipo_retorno nome(tipo parâmetro1, ..., tipo parâmetroN);
        \end{minted}

        \item O protótipo de uma função deve aparecer antes do seu uso. 
        \item Em geral coloca-se os protótipos de funções no início do seu arquivo do programa.
    \end{itemize}
\end{frame}

%%%%%%%%%%%%%%%%%%%%%%%%%%%%%%%%%%%%%%%%%%%%%%
\begin{frame}[fragile]{Protótipo de funções: declarando uma função sem defini-la}

    Em geral o programa é organizado da seguinte forma:

    \begin{minted}[fontsize=\scriptsize]{c}
        #include <stdio.h>
        #include <outras bibliotecas>
        
        /* Protótipos de funções */
        int fun1(lista de parâmetros);

        int main() {
            comandos;
        }

        /* Declarações de funções */
        int fun1(lista de parâmetros) {
            comandos;
        }
    \end{minted}
\end{frame}

%%%%%%%%%%%%%%%%%%%%%%%%%%%%%%%%%%%%%%%
\begin{frame}[fragile]{Protótipo de Funções: Exemplo 1}

    \begin{minted}[fontsize=\scriptsize]{c}
        #include <stdio.h>

        float soma(float op1, float op2);
        float subtrai(float op1, float op2);

        int main() {
            float a = 0, b = 5;
            printf("soma = %f\nsubtracao = %f\n", soma (a, b), subtrai(a, b));
            return 0;
        }

        float soma(float op1, float op2) {
            return op1 + op2;
        }

        float subtrai(float op1, float op2) {
            return op1 - op2;
        }
    \end{minted}
\end{frame}

%%%%%%%%%%%%%%%%%%%%%%%%%%%%%%%%%%%%%%%
%%%%%%%%%%%%%%%%%%%%%%%%%%%%%%%%%%%%%%%
%%%%%%%%%%%%%%%%%%%%%%%%%%%%%%%%%%%%%%%
%%%%%%%%%%%%%%%%%%%%%%%%%%%%%%%%%%%%%%%
%%%%%%%%%%%%%%%%%%%%%%%%%%%%%%%%%%%%%%%
%%%%%%%%%%%%%%%%%%%%%%%%%%%%%%%%%%%%%%%

\section{Funções Podem Invocar Funções}

%%%%%%%%%%%%%%%%%%%%%%%%%%%%%%%%%%%%%%%
\begin{frame}[fragile]{Funções Podem Invocar Funções}

    \begin{itemize}
        \item Nos exemplos anteriores apenas a função {\bf main} invocava funções por nós definidas.
        \item Isto não é uma regra. Qualquer função pode invocar outra função (exceto a {\bf main}, que é invocada apenas pelo sistema operacional).
        \item Veja o exemplo no próximo slide.
    \end{itemize}
\end{frame}

%%%%%%%%%%%%%%%%%%%%%%%%%%%%%%%%%%%%%%%
\begin{frame}[fragile]{Funções Podem Invocar Funções}

    O que será impresso?

            \begin{minted}[fontsize=\tiny]{c}
        #include <stdio.h>

        int fun1(int a);
        int fun2(int b);

        int main() {
            int c = 5;
            c = fun1(c);
            printf("c = %d\n", c);
            return 0;
        }

        int fun1(int a) {
            a = a + 1;
            a = fun2(a);
            return a;
        }

        int fun2(int b) {
            b = 2 * b;
            return b;
        }
    \end{minted}
\end{frame}

%%%%%%%%%%%%%%%%%%%%%%%%%%%%%%%%%%%%%%%%%%%%%%
%%%%%%%%%%%%%%%%%%%%%%%%%%%%%%%%%%%%%%%%%%%%%%
%%%%%%%%%%%%%%%%%%%%%%%%%%%%%%%%%%%%%%%%%%%%%%
%%%%%%%%%%%%%%%%%%%%%%%%%%%%%%%%%%%%%%%%%%%%%%
%%%%%%%%%%%%%%%%%%%%%%%%%%%%%%%%%%%%%%%%%%%%%%
%%%%%%%%%%%%%%%%%%%%%%%%%%%%%%%%%%%%%%%%%%%%%%

\section{Escopo de Variáveis: variáveis locais e globais}

%%%%%%%%%%%%%%%%%%%%%%%%%%%%%%%%%%%%%%%%%%%%%%
\begin{frame}{Variáveis locais e variáveis globais}

    \begin{itemize}[<+->]
        \item Uma variável é chamada \alert{local} se ela foi declarada dentro de uma função.
        Nesse caso ela existe somente dentro da função e, após o término da execução desta, a variável deixa de existir.
        {\bf Variáveis parâmetros também são variáveis locais.}
        \item Uma variável é chamada \alert{global} se ela for declarada fora de qualquer função.
        Essa variável é visível em todas as funções.
        Qualquer função pode alterá-la e ela existe durante toda a execução do programa.
    \end{itemize}
\end{frame}

%%%%%%%%%%%%%%%%%%%%%%%%%%%%%%%%%%%%%%%%%%%%%%
\begin{frame}[fragile]{Organização de um programa}

    Em geral um programa é organizado da seguinte forma:

    \begin{minted}[fontsize=\scriptsize]{c}
        #include <stdio.h>
        #include <outras bibliotecas>

        protótipos de funções
        
        declaração de Variáveis Globais

        int main() {
            declaração de variáveis locais à main
            comandos;
        }

        int fun1(parâmetros) { /* parâmetros também são variáveis locais */
            declaração de variáveis locais à fun1
            comandos;
        }

        ...
    \end{minted}
\end{frame}

%%%%%%%%%%%%%%%%%%%%%%%%%%%%%%%%%%%%%%%%%%%%%%
\begin{frame}{Escopo de variáveis}

    \begin{itemize}
        \item O \alert{escopo} de uma variável determina de quais partes do código ela pode ser acessada, ou seja, de quais partes do código a variável é visível.
        \item A regra de escopo em C é bem simples:
        \begin{itemize}
            \item As variáveis globais são visíveis por todas as funções.
            \item As variáveis locais são visíveis apenas na função onde foram declaradas.
        \end{itemize}
    \end{itemize}
\end{frame}

%%%%%%%%%%%%%%%%%%%%%%%%%%%%%%%%%%%%%%%%%%%%%%
\begin{frame}[fragile]{Escopo de variáveis}

    \begin{minted}[fontsize=\tiny]{c}
        #include <stdio.h>

        void fun1();
        int fun2(int local_b);

        int global;

        int main() {
            int local_main;
            /* Neste ponto são visíveis global e local_main */
        }

        void fun1() {
            int local_a;
            /* Neste ponto são visíveis global e local_a */
        }
    
        int fun2(int local_b) {
            int local_c;
            /* Neste ponto são visíveis global, local_b e local_c */
        }
    \end{minted}
\end{frame}

%%%%%%%%%%%%%%%%%%%%%%%%%%%%%%%%%%%%%%%%%%%%%%
\begin{frame}[fragile]{Escopo de variáveis}

    \begin{itemize}
        \item É possível declarar variáveis locais com o mesmo nome de variáveis globais.
        \item Nesta situação, a variável local ``esconde'' a variável global.

        \begin{minted}[fontsize=\tiny]{c}
            #include <stdio.h>
            
            void fun();

            int nota = 10;

            int main() {
                nota = 20;
                fun();
                printf("%d\n", nota);
                return 0;
            }

            void fun() {
                int nota;
                nota = 5;
                /* Neste ponto nota é a variável local de fun. */
            }
        \end{minted}
    \end{itemize}
\end{frame}

%%%%%%%%%%%%%%%%%%%%%%%%%%%%%%%%%%%%%%%%%%%%%%
\begin{frame}[fragile]{Exemplo 1}

    \begin{minted}[fontsize=\tiny]{c}
        #include <stdio.h>

        void fun1();
        void fun2();

        int x;
        
        int main() {
            x = 1;
            fun1();
            fun2();
            printf("main: %d\n", x);
            return 0;
        }

        void fun1() {
            x = x + 1;
            printf("fun1: %d\n", x);
        }

        void fun2() {
            int x = 3;
            printf("fun2: %d\n", x);
        }
    \end{minted}
    
    O que será impresso?
\end{frame}

%%%%%%%%%%%%%%%%%%%%%%%%%%%%%%%%%%%%%%%%%%%%%%
\begin{frame}[fragile]{Exemplo 2}

    \begin{minted}[fontsize=\tiny]{c}
        #include <stdio.h>
        
        void fun1();
        void fun2();

        int x = 1;
        
        int main() {
            int x = 1;
            fun1();
            fun2();
            printf("main: %d\n", x);
            return 0;
        }

        void fun1() {
            x = x + 1;
            printf("fun1: %d\n", x);
        }

        void fun2() {
            int x = 4;
            printf("fun2: %d\n", x);
        }
    \end{minted}

    O que será impresso?
\end{frame}

%%%%%%%%%%%%%%%%%%%%%%%%%%%%%%%%%%%%%%%%%%%%%%
\begin{frame}[fragile]{Exemplo 3}

    \begin{minted}[fontsize=\tiny]{c}
        #include <stdio.h>

        void fun1();
        void fun2(int x);

        int x = 1;

        int main() {
            x = 2;
            fun1();
            fun2(x);
            printf("main: %d\n", x);
            return 0;
        }

        void fun1() {
            x = x + 1;
            printf("fun1: %d\n", x);
        }

        void fun2(int x) {
            x = x + 1;
            printf("fun2: %d\n", x);
        }
    \end{minted}

    O que será impresso?
\end{frame}

%%%%%%%%%%%%%%%%%%%%%%%%%%%%%%%%%%%%%%%%%%%%%%
\begin{frame}[fragile]{Variáveis locais e variáveis globais}

    \begin{itemize}
        \item O uso de variáveis globais deve ser evitado pois é uma causa comum de erros:
        \begin{itemize}
            \item Partes distintas e funções distintas podem alterar a variável global, causando uma grande interdependência entre estas partes distintas de código.
        \end{itemize}
        \item A legibilidade do seu código também pode piorar com o uso de variáveis globais:
        \begin{itemize}
            \item Ao ler uma função que usa uma variável global é difícil inferir seu valor inicial e portanto qual o resultado da função sobre a variável global.
        \end{itemize}
    \end{itemize}
\end{frame}

%%%%%%%%%%%%%%%%%%%%%%%%%%%%%%%%%%%%%%%
%%%%%%%%%%%%%%%%%%%%%%%%%%%%%%%%%%%%%%%
%%%%%%%%%%%%%%%%%%%%%%%%%%%%%%%%%%%%%%%
%%%%%%%%%%%%%%%%%%%%%%%%%%%%%%%%%%%%%%%
%%%%%%%%%%%%%%%%%%%%%%%%%%%%%%%%%%%%%%%
%%%%%%%%%%%%%%%%%%%%%%%%%%%%%%%%%%%%%%%

\section{Exemplo utilizando funções}

%%%%%%%%%%%%%%%%%%%%%%%%%%%%%%%%%%%%%%%
\begin{frame}[fragile]{Exemplo utilizando funções}

    Em aulas anteriores vimos como testar se um número em {\bf candidato} é primo:

    \begin{minted}[fontsize=\scriptsize]{c}
        divisor = 2;
        eh_primo = 1;
        while (divisor <= candidato/2) {
            if (candidato % divisor == 0) {
                eh_primo = 0;
                break;
            }
            divisor++;
        }
        if (eh_primo)
            printf("%d, ", candidato);
    \end{minted} 

    E usamos isso para imprimir os $n$ primeiros números primos:
\end{frame}

%%%%%%%%%%%%%%%%%%%%%%%%%%%%%%%%%%%%%%%
\begin{frame}[fragile]{Exemplo utilizando funções}
    \begin{minted}[fontsize=\tiny]{c}
        int main() {
            int divisor, n, eh_primo, candidato, primosImpressos;
            scanf("%d", &n);
            if (n >= 1) {
                printf("2, ");
                primosImpressos = 1;
                candidato = 3;
                while (primosImpressos < n) {
                    divisor = 2;
                    eh_primo = 1;
                    while (divisor <= candidato/2) {
                        if (candidato % divisor == 0) {
                            eh_primo = 0;
                            break;
                        }
                        divisor++;
                    }
                    if (eh_primo) {
                        printf("%d, ", candidato);
                        primosImpressos++;
                    }
                    candidato = candidato+2; 
                }
            }
            return 0;
        }
    \end{minted} 
\end{frame}

%%%%%%%%%%%%%%%%%%%%%%%%%%%%%%%%%%%%%%%
\begin{frame}[fragile]{Exemplo utilizando funções}

    \begin{itemize}
        \item Podemos criar uma função que testa se um número é primo ou não (note que isto é exatamente um bloco logicamente bem definido).
        \item Depois fazemos chamadas para esta função.
    \end{itemize}

    \begin{minted}[fontsize=\scriptsize]{c}
        int eh_primo(int candidato) {
            int divisor = 2;

            while (divisor <= candidato/2) {
                if (candidato % divisor == 0) {
                    return 0;
                }
                divisor++;
            }

            return 1;
        }
    \end{minted} 
\end{frame}

%%%%%%%%%%%%%%%%%%%%%%%%%%%%%%%%%%%%%%%
\begin{frame}[fragile]{Exemplo utilizando funções}

    \begin{minted}[fontsize=\tiny]{c}
        #include <stdio.h>
        
        int eh_primo(int candidato); /* retorna 1 se candidato é primo, e 0 caso contrário */

        int main() {
            int n, candidato, primosImpressos;

            scanf("%d", &n);

            if (n >= 1) {
                printf("2, ");
                primosImpressos = 1;
                candidato = 3;
                while (primosImpressos < n) {
                    if (eh_primo(candidato)) {
                        printf("%d, ", candidato);
                        primosImpressos++;
                    }
                    candidato = candidato + 2;
                }
            }

            return 0;
        }
    \end{minted}
\end{frame}

%%%%%%%%%%%%%%%%%%%%%%%%%%%%%
%%%%%%%%%%%%%%%%%%%%%%%%%%%%%
%%%%%%%%%%%%%%%%%%%%%%%%%%%%%
%%%%%%%%%%%%%%%%%%%%%%%%%%%%%
%%%%%%%%%%%%%%%%%%%%%%%%%%%%%
%%%%%%%%%%%%%%%%%%%%%%%%%%%%%

\section{Exercícios}

%%%%%%%%%%%%%%%%%%%%%%%%%%%%%%%%%%%%%%%
\begin{frame}[fragile]{Exercício}

    Escreva uma função que computa a potência $a^b$ para valores $a$ (double) e $b$ (int) passados por parâmetro (não use bibliotecas como math.h).
    Sua função deve ter o seguinte protótipo:

    \begin{minted}{c}
        double pot(double a, int b);
    \end{minted}

    Use a função anterior e crie um programa que imprima todas as potências: $$2^0, 2^1, \ldots, 2^{10}, 3^0, \ldots, 3^{10}, \ldots, 10^{10}.$$
\end{frame}

%%%%%%%%%%%%%%%%%%%%%%%%%%%%%%%%%%%%%%%
\begin{frame}[fragile]{Exercício}

    Escreva uma função que computa o fatorial de um número $n$ passado por parâmetro.
    Sua função deve ter o seguinte protótipo:

    \begin{minted}{c}
        long int fat(int n);
    \end{minted}
    OBS: Caso $n \leq 0$, seu programa deve retornar 1.

    Use a função anterior e crie um programa calcula o coeficiente binomial de dois números $n$ e $k$ lidos:
    $${n \choose k} = \frac{n!}{k! (n-k)!}$$

    \alert{Pergunta: o que acontece se você define a função como sendo do tipo \cod{int}?}
\end{frame}

%%%%%%%%%%%%%%%%%%%%%%%%%%%%%
\begin{frame}[fragile]{Exercício} 

    Escreva uma função em C para computar a raiz quadrada de um número positivo.
    Use a idéia abaixo, baseada no método de aproximações sucessivas de Newton descrito abaixo.

    Seja $Y$ um número. Sua raiz quadrada é raiz da equação $$f(x) = x^2 - Y \enspace .$$
    A primeira aproximação é $x_1 = Y/2$.
    A $(n+1)$-ésima aproximação é
    $$x_{n+1} = x_n -  \frac{f(x_n)}{f'(x_n)}$$

    A função deverá retornar o valor da vigésima aproximação.
\end{frame}

%%%%%%%%%%%%%%%%%%%%%%%%%%%%%
\begin{frame}[fragile]{Exercício}
    
    Faça um programa que teste quais números de uma sequência fazem parte da sequência de Fibonacci.

    O programa deve receber inicialmente um inteiro $n$. Em seguida, deve receber $n$ números inteiros positivos.

    A resposta consiste de uma única linha, com os números da sequência da entrada que fazem parte da sequência de Fibonacci separados por um espaço.

    Por exemplo, se a entrada é ``\texttt{7 3 9 -7 4 1 0 3}'', então a saída é ``\texttt{3 1 0 3}''.
\end{frame}

%%%%%%%%%%%%%%%%%%%%%%%%%%%%%
\begin{frame}[fragile]{Exercício}
    
    Escreva um programa que mostre os dois números primos mais próximos de um dado número $n$.
    
    Por exemplo, se $n = 24$, os dois números primos mais próximos dele são $23$ e $29$.
    Se $n = 47$, então os dois números primos mais próximos dele são $47$ e $53$.
\end{frame}

\end{document}
