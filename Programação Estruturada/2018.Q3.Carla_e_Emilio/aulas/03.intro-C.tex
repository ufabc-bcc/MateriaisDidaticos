%    Copyright © 2015
%  Eduardo Candido Xavier <eduardo@ic.unicamp.br>
%
%  This work is free. You can redistribute it and/or modify it under the
%  terms of the Do What The Fuck You Want To Public License, Version 2,
%  as published by Sam Hocevar. See the COPYING file for more details.
%
%  DO WHAT THE FUCK YOU WANT TO PUBLIC LICENSE
%                   Version 2, December 2004
%
%   Copyright (C) 2004 Sam Hocevar <sam@hocevar.net>
%
%   Everyone is permitted to copy and distribute verbatim or modified
%   copies of this license document, and changing it is allowed as long
%   as the name is changed.
%
%  DO WHAT THE FUCK YOU WANT TO PUBLIC LICENSE
%   TERMS AND CONDITIONS FOR COPYING, DISTRIBUTION AND MODIFICATION
%
%      0. You just DO WHAT THE FUCK YOU WANT TO.

\documentclass[handout]{beamer}
\usetheme{metropolis}
\beamertemplatetransparentcoveredhigh

\usepackage[portuges]{babel}
\usepackage{graphicx}
\graphicspath{{./figs/}}
\usepackage{listings}
\usepackage{color}
\usepackage{hyperref}
\usepackage{xpatch}
\usepackage[outputdir=build]{minted}

\makeatletter
\AtBeginEnvironment{minted}{\dontdofcolorbox}
\def\dontdofcolorbox{\renewcommand\fcolorbox[4][]{##4}}
\xpatchcmd{\inputminted}{\minted@fvset}{\minted@fvset\dontdofcolorbox}{}{}
\xpatchcmd{\mintinline}{\minted@fvset}{\minted@fvset\dontdofcolorbox}{}{}
\makeatother
\setminted[c]{
  linenos=true,
  breaklines=true,
  encoding=utf8,
  frame=lines,
  framerule=0.5pt,
  autogobble,
  fontsize=\small,
}
\setminted[bash]{
  linenos=true,
  encoding=utf8,
  frame=lines,
  framerule=0.5pt,
  autogobble,
  fontsize=\small
}

\newcommand{\cod}[1]{\mintinline{c}{#1}}


\definecolor{dkgreen}{rgb}{0,0.6,0}
\definecolor{gray}{rgb}{0.5,0.5,0.5}
\definecolor{mauve}{rgb}{0.58,0,0.82}


\definecolor{Purple}{HTML}{911146}
\definecolor{Orange}{HTML}{CF4A30}
\setbeamercolor{alerted text}{fg=Orange}
\setbeamercolor{frametitle}{bg=Purple}
\setbeamercolor{block body}{bg=Purple!20,fg=black}
\setbeamercolor{block title}{bg=Purple!50,fg=black}
\setbeamertemplate{blocks}[rounded][shadow=true]


\newcommand{\setcoverbg}{
  \setbeamertemplate{background}
  {\includegraphics[width=\paperwidth,height=\paperheight]{backgrounds/coverbg}}
}
\newcommand{\setsectionbg}{
  \setbeamertemplate{background}
  {\includegraphics[width=\paperwidth,height=\paperheight]{backgrounds/blank}}
}

\title{Programação Estruturada}
\subtitle{Introdução à linguagem C}

\author{Professores Emílio Francesquini e Carla Negri Lintzmayer}
\institute{Centro de Matemática, Computação e Cognição\\ Universidade Federal do ABC}
\date{2018.Q3}

\begin{document}

\setcoverbg
\maketitle
\setsectionbg


%%%%%%%%%%%%%%%%%%%%%%%%%%%%%%%%
%%%%%%%%%%%%%%%%%%%%%%%%%%%%%%%%
%%%%%%%%%%%%%%%%%%%%%%%%%%%%%%%%
%%%%%%%%%%%%%%%%%%%%%%%%%%%%%%%%
%%%%%%%%%%%%%%%%%%%%%%%%%%%%%%%%
%%%%%%%%%%%%%%%%%%%%%%%%%%%%%%%%
%%%%%%%%%%%%%%%%%%%%%%%%%%%%%%%%
\section{Programação estruturada}

%%%%%%%%%%%%%%%%%%%%%%%%%%%%%%%%
\begin{frame}{O que é programação estruturada}

    Técnica/paradigma de programação que visa o desenvolvimento de algoritmos por meio de estruturas de controle visando clareza e qualidade
    \begin{itemize}
        \item Estrutura sequencial
        \item Estrutura condicional
        \item Estrutura de repetição
        \item Modularização
    \end{itemize}

    Exemplos de linguagens que suportam o paradigma estruturado: Pascal, Ada, Java, C, C++.
\end{frame}



%%%%%%%%%%%%%%%%%%%%%%%%%%%%%%%%
%%%%%%%%%%%%%%%%%%%%%%%%%%%%%%%%
%%%%%%%%%%%%%%%%%%%%%%%%%%%%%%%%
%%%%%%%%%%%%%%%%%%%%%%%%%%%%%%%%
%%%%%%%%%%%%%%%%%%%%%%%%%%%%%%%%
%%%%%%%%%%%%%%%%%%%%%%%%%%%%%%%%
%%%%%%%%%%%%%%%%%%%%%%%%%%%%%%%%
\section{A linguagem C}

%%%%%%%%%%%%%%%%%%%%%%%%%%%%%%%%
\begin{frame}{Sobre a linguagem C}

    \begin{itemize}[<+->]
        \item Criada em 1972 por Dennis Ritchie
        \begin{itemize}
            \item The C Programming Language, Kernighan e Ritchie
        \end{itemize}
        \item Linguagem compilada, de propósito geral, estruturada, imperativa, procedural
        \item Várias especificações (\textbf{ANSI}, C89, C99, C11)
        \begin{itemize}
            \item Neste curso nos concentraremos na especificação ANSI.
        \end{itemize}
    \end{itemize}
\end{frame}


%%%%%%%%%%%%%%%%%%%%%%%%%%%%%%%%
\begin{frame}{Programas em C}

    \begin{itemize}[<+->]
        \item Um programa em C é conjunto de um ou mais arquivos textos, contendo declarações e operações da linguagem (código fonte)
        \item Consiste em uma ou mais \textit{funções}
        \item A única obrigatória é denominada \textbf{main()}
    \end{itemize}
\end{frame}


%%%%%%%%%%%%%%%%%%%%%%%%%%%%%%%%
\begin{frame}[fragile]{Programas em C}

    \begin{minted}{c}
        #include <stdio.h>

        int main() {
            printf("Hello world!\n");
            return 0;
        }
    \end{minted}
\end{frame}


%%%%%%%%%%%%%%%%%%%%%%%%%%%%%%%%
%%%%%%%%%%%%%%%%%%%%%%%%%%%%%%%%
%%%%%%%%%%%%%%%%%%%%%%%%%%%%%%%%
%%%%%%%%%%%%%%%%%%%%%%%%%%%%%%%%
%%%%%%%%%%%%%%%%%%%%%%%%%%%%%%%%
%%%%%%%%%%%%%%%%%%%%%%%%%%%%%%%%
%%%%%%%%%%%%%%%%%%%%%%%%%%%%%%%%
\section{Variáveis}

%%%%%%%%%%%%%%%%%%%%%%%%%%%%%%%%%%%%%%%%%%%%%%%%%%%%%%%
\begin{frame}{Variáveis}

    \begin{block}{Definição}
        Variáveis são locais onde armazenamos valores.

        Toda variável é caracterizada por um \textbf{nome}, que a identifica em um programa, e por um \textbf{tipo}, que determina o que pode ser armazenado naquela variável.
    \end{block}

    \pause
    \begin{itemize}[<+->]
        \item Durante a execução do programa, um pedacinho da memória corresponde às variáveis
        \item Em C, existem 5 tipos básicos de variáveis: \texttt{char}, \texttt{int}, \texttt{float}, \texttt{double} e \texttt{void}\footnote{Estritamente falando, \texttt{void} não é um tipo, mas sim uma palavra reservada utilizada para marcar a ausência dos outros tipos}.
    \end{itemize}
\end{frame}

%%%%%%%%%%%%%%%%%%%%%%%%%%%%%%%%%%%%%%%%%%%%%%%%%%%%%%%
\begin{frame}{Declarando uma variável em C}

    \begin{block}{Comando de declaração}
        \texttt{\textbf{tipo\_variável nome\_variável;}}
    \end{block}

    \pause
    Exemplos corretos:
    \begin{itemize}
        \item \texttt{int soma;}
        \item \texttt{float preco\_abacaxi;}
        \item \texttt{char resposta;}
    \end{itemize}

    \pause
    Exemplos incorretos (por quê?):
    \begin{itemize}
        \item \texttt{soma int;}
        \item \texttt{float preco\_abacaxi}
    \end{itemize}
\end{frame}

%%%%%%%%%%%%%%%%%%%%%%%%%%%%%%%%%%%%%%%%%%%%%%%%%%%%%%%
\begin{frame}{Variáveis inteiras}

    \begin{block}{Variáveis inteiras}
        Utilizadas para armazenar valores inteiros, como $13$ ou $1102$ ou $24$.
    \end{block}

    \pause
    Abaixo temos os {\bf tipos da linguagem C} que servem para armazenar inteiros:
    \begin{description}
        \item[int] Inteiro cujo comprimento depende do processador.
        É o mais utilizado.
        Em processadores de 32 bits (4 bytes) pode armazenar valores entre -2.147.483.648 e 2.147.483.647.

        \pause
        \item[unsigned int] Inteiro cujo comprimento depende do processador e que armazena somente valores positivos.
        Em processadores de 32 bits pode armazenar valores entre 0 e 4.294.967.295.
    \end{description}

    \pause
    \alert{Pergunta:} Como fica o caso dos processadores de 64 bits?
\end{frame}

%%%%%%%%%%%%%%%%%%%%%%%%%%%%%%%%%%%%%%%%%%%%%%%%%%%%%%%
\begin{frame}{Variáveis inteiras}

    \begin{description}[<+->]
        \item[long int] Inteiro que ocupa 64 bits (8 bytes) em computadores Intel de 64 bits, e pode armazenar valores entre aprox. $-9\times10^{18}$ e aprox. $9\times10^{18}$
        \item[unsigned long int] Inteiro que ocupa 64 bits em computadores Intel de 64 bits, e armazena valores entre $0$ e aprox. $18 \times 10^{18}$
        \item[short int] Inteiro que ocupa 16 bits (2 bytes) e pode armazenar valores entre -32.768 e 32.767
        \item[unsigned short int] Inteiro que ocupa 16 bits e pode armazenar valores entre 0 e 65.535
    \end{description}
\end{frame}

%%%%%%%%%%%%%%%%%%%%%%%%%%%%%%%%%%%%%%%%%%%%%%%%%%%%%%%
\begin{frame}{Variáveis inteiras}

    Exemplos de declaração de variáveis inteiras:
    \begin{itemize}
        \item \texttt{int numVoltas;}
        \item \texttt{int ano;}
        \item \texttt{unsigned int quantidadeChapeus;}
    \end{itemize}

    \pause
    Exemplos inválidos:
    \begin{itemize}
        \item \texttt{int int numVoltas;}
        \item \texttt{unsgned int ano;}
    \end{itemize}
\end{frame}

%%%%%%%%%%%%%%%%%%%%%%%%%%%%%%%%%%%%%%%%%%%%%%%%%%%%%%%
\begin{frame}{Declarando várias variáveis}
    
    Você pode declarar várias variáveis de um mesmo tipo em uma linha.

    Basta separar as variáveis por vírgula:
    \begin{itemize}
        \item \texttt{int numVoltas, ano;}
        \item \texttt{unsigned int val1, val2, val3, val4;}
    \end{itemize}
\end{frame}

%%%%%%%%%%%%%%%%%%%%%%%%%%%%%%%%%%%%%%%%%%%%%%%%%%%%%%%
\begin{frame}{Variáveis do tipo caractere}

    \begin{block}{Variáveis caracteres}
        Utilizadas para armazenar uma letra ou outro símbolo existente em textos.
        Ex: \texttt{'a'}, \texttt{'3'}, \texttt{','}, \texttt{'+'}.
    \end{block}

    \pause
    Exemplos de declaração:
    \begin{itemize}
        \item \texttt{char umaLetra;}
        \item \texttt{char S\_ou\_N;}
    \end{itemize}
\end{frame}

%%%%%%%%%%%%%%%%%%%%%%%%%%%%%%%%%%%%%%%%%%%%%%%%%%%%%%%
\begin{frame}{Variáveis de tipo ponto flutuante}

    \begin{block}{Variáveis de ponto flutuante}
        Armazenam valores reais, mas possuem problemas de precisão pois há uma quantidade limitada de memória para armazenar um número real.
        Ex: 2.1345 ou 9098.123.
    \end{block}

    \pause
    Abaixo temos os {\bf tipos da linguagem C} que servem para armazenar números de ponto flutuante:
    \begin{description}[<+->]
        \item[float] Utiliza 32 bits, e na prática tem precisão de aproximadamente 6 casas decimais (depois do ponto). Pode armazenar valores entre $3.4 \times 10^{-38}$ e $3.4 \times 10^{38}$
        \item[double] Utiliza 64 bits, e na prática tem precisão de aproximadamente 15 casas decimais. Pode armazenar valores entre $1.7 \times 10^{-308}$ e $1.7 \times 10^{308}$
    \end{description}
\end{frame}

%%%%%%%%%%%%%%%%%%%%%%%%%%%%%%%%%%%%%%%%%%%%%%%%%%%%%%%
\begin{frame}{Variáveis de tipo ponto flutuante}

    Exemplos de declaração:
    \begin{itemize}
        \item \texttt{float salario, resultado;}
        \item \texttt{double cotacaoDolar;}
    \end{itemize}
\end{frame}

%%%%%%%%%%%%%%%%%%%%%%%%%%%%%%%%%%%%%%%%%%%%%%%%%%%%%%%
\begin{frame}[fragile]{Regras para nome de variável em C}

    \begin{itemize}[<+->]
        \item \textbf{Deve} começar com uma letra (maiúscula ou minúscula) ou subscrito (\_)
        \item \textbf{Nunca} pode começar com um número
        \item Pode conter letras maiúsculas, minúsculas, números e subscrito
        \item Não pode conter os símbolos \texttt{\{} \texttt{(} \texttt{+} \texttt{-} \texttt{*} \texttt{/} \texttt{\textbackslash} \texttt{;} \texttt{.} \texttt{,} \texttt{?}
        \item Letras maiúsculas e minúsculas são diferentes:
        \begin{minted}{c}
            int c, C;
        \end{minted}
    \end{itemize}
\end{frame}

%%%%%%%%%%%%%%%%%%%%%%%%%%%%%%%%%%%%%%%%%%%%%%%%%%%%%%%
\begin{frame}[fragile]{Regras para nome de variável em C}

    As seguintes palavras já tem um significado na linguagem C (são reservadas) e por esse motivo não podem ser utilizadas como nome de variáveis:
    
    \begin{verbatim}
    auto        double      int         struct
    break       else        long        switch
    case        enum        register    typedef
    char        extern      return      union
    const       float       short       unsigned
    continue    for         signed      void
    default     goto        sizeof      volatile
    do          if          static      while
    \end{verbatim}

\end{frame}

%%%%%%%%%%%%%%%%%%%%%%%%%%%%%%%%%%%%%%%%%%%%%%%%%%%%%%%
%%%%%%%%%%%%%%%%%%%%%%%%%%%%%%%%%%%%%%%%%%%%%%%%%%%%%%%
%%%%%%%%%%%%%%%%%%%%%%%%%%%%%%%%%%%%%%%%%%%%%%%%%%%%%%%
%%%%%%%%%%%%%%%%%%%%%%%%%%%%%%%%%%%%%%%%%%%%%%%%%%%%%%%
%%%%%%%%%%%%%%%%%%%%%%%%%%%%%%%%%%%%%%%%%%%%%%%%%%%%%%%
%%%%%%%%%%%%%%%%%%%%%%%%%%%%%%%%%%%%%%%%%%%%%%%%%%%%%%%

\section{Atribuição}

%%%%%%%%%%%%%%%%%%%%%%%%%%%%%%%%%%%%%%%%%%%%%%%%%%%%%%%
\begin{frame}[fragile]{Atribuindo valores a variáveis}

    \begin{block}{Comando de atribuição}
        \texttt{\textbf{nome\_variável = valor;}}

        ou

        \texttt{\textbf{nome\_variável = expressão;}}
    \end{block}

    \pause
    Exemplos:
    \begin{minted}{c}
        int a;
        float c;
        a = 5;
        c = 67.89505456;
        a = 3 + 5;
    \end{minted}

    \pause
    Atribuir um valor de uma expressão para uma variável significa calcular o valor daquela expressão e então copiar aquele valor para a variável (nessa ordem).
\end{frame}

%%%%%%%%%%%%%%%%%%%%%%%%%%%%%%%%%%%%%%%%%%%%%%%%%%%%%%%%%
\begin{frame}{Atribuição}

    \begin{itemize}
        \item O sinal de igual (\texttt{=}) no comando de atribuição é chamado de {\bf operador de atribuição}
        \pause
        \item Veremos outros operadores mais adiante
    \end{itemize}

    \bigskip
    
    \pause
    \begin{minipage}{0.4\textwidth}
        À esquerda do operador de atribuição deve existir somente o nome de uma \textbf{variável}
    \end{minipage}
    \hfill = \hfill
    \pause
    \begin{minipage}{0.4\textwidth}
        À direita, deve haver uma \textbf{expressão} cujo valor será calculado e armazenado na variável
    \end{minipage}
\end{frame}

%%%%%%%%%%%%%%%%%%%%%%%%%%%%%%%%%%%%%%
\begin{frame}[fragile]{Variáveis e constantes}

    \begin{block}{Constantes}
        São valores previamente determinados e que, por algum motivo, devem aparecer dentro de um programa
    \end{block}

    \pause
    \begin{itemize}[<+->]
        \item Constantes também possuem um tipo
        \item Os tipos permitidos são exatamente os mesmos das variáveis, mais o tipo {\tt string}, que corresponde a uma sequência de caracteres
        \item Exemplos de constantes:
        \begin{minted}{c}
            85, 0.10, 'c', "Hello, world!"
        \end{minted}
    \end{itemize}
\end{frame}

%%%%%%%%%%%%%%%%%%%%%%%%%%%%%%%%%%%%%%%%%%%%%%%%%%%%%%%%%
\begin{frame}[fragile]{Expressões simples}

    \begin{block}{}
        Uma constante é uma expressão e, como tal, pode ser atribuída a uma variável (ou ser usada em qualquer outro lugar onde uma expressão seja válida)
    \end{block}

    \pause
    Exemplos:
    \begin{minted}{c}
        int a;
        a = 10;
        char b;
        b = 'F';
        double c;
        c = 3.141592;
    \end{minted}

\end{frame}

%%%%%%%%%%%%%%%%%%%%%%%%%%%%%%%%%%%%%%%%%%%%%%%%%%%%%%%%%
\begin{frame}[fragile]{Expressões simples}

    \begin{block}{}
        Uma variável também é uma expressão e pode ser atribuída a outra variável
    \end{block}

    \pause
    Exemplo:
    \begin{minted}{c}
        int a, b;
        a = 5;
        b = a;
    \end{minted}
\end{frame}


%%%%%%%%%%%%%%%%%%%%%%%%%%%%%%%%%%%%%%%%%%%%%%%%%%%%%%%%%
\begin{frame}[fragile]{Expressões simples}

    \begin{minted}{c}
        int a, b, c;

        a = 5 + 5 + 10;
        b = 6;
        c = a + b;
        a = c + 4;
    \end{minted}

    Qual é o valor de \texttt{a}, \texttt{b} e \texttt{c}?

\end{frame}

%%%%%%%%%%%%%%%%%%%%%%%%%%%%%%%%%%%%%%%%%%%%%%%%%%%%%%%%%
\begin{frame}[fragile]{Exemplos de atribuição}

    A declaração de uma variável \textbf{sempre} deve ocorrer antes de seu uso

    \begin{minted}{c}
        int a, b;
        float f;
        char h;

        a = 10;
        b = -15;
        f = 10.0;
        h = 'A';

        a = b;
        f = a;
        a = (b + a);
    \end{minted}

    Qual o valor final na variável \texttt{a}?

\end{frame}

%% TODO falar sobre *inicialização*

%%%%%%%%%%%%%%%%%%%%%%%%%%%%%%%%%%%%%%%%%%%%%%%%%%%%%%%%%
\begin{frame}[fragile]{Exemplos errados de atribuição}

    \begin{minted}{c}
        int a, b;
        float f, g;
        char h;

        a b = 10;
        b = -15
        d = 90;
    \end{minted}

    Quais são os erros?
\end{frame}

%%%%%%%%%%%%%%%%%%%%%%%%%%%%%%%%%%%%%%%%%%%%%%%%%%%%%%%%%
%%%%%%%%%%%%%%%%%%%%%%%%%%%%%%%%%%%%%%%%%%%%%%%%%%%%%%%%%
%%%%%%%%%%%%%%%%%%%%%%%%%%%%%%%%%%%%%%%%%%%%%%%%%%%%%%%%%
%%%%%%%%%%%%%%%%%%%%%%%%%%%%%%%%%%%%%%%%%%%%%%%%%%%%%%%%%
%%%%%%%%%%%%%%%%%%%%%%%%%%%%%%%%%%%%%%%%%%%%%%%%%%%%%%%%%
%%%%%%%%%%%%%%%%%%%%%%%%%%%%%%%%%%%%%%%%%%%%%%%%%%%%%%%%%
%%%%%%%%%%%%%%%%%%%%%%%%%%%%%%%%%%%%%%%%%%%%%%%%%%%%%%%%%

\section{Estrutura de um programa em C}
%%%%%%%%%%%%%%%%%%%%%%%%%%%%%%%%%%%%%%%%%%%%%%%%%%%%%%%%%

\begin{frame}[fragile]{Estrutura básica de um programa em C}

    \begin{minted}{c}
        Declaracao de bibliotecas usadas

        Declaracao de variaveis

        int main() {
            Declaracao de variaveis

            Comandos

            return 0;
        }
    \end{minted}

\end{frame}

%%%%%%%%%%%%%%%%%%%%%%%%%%%%%%%%%%%%%%%%%%%%%%%%%%%%%%%%%
\begin{frame}[fragile]{Estrutura básica de um programa em C}

    \begin{minted}{c}
        #include <stdio.h>

        int main() {
            int a;
            int b, c;

            a = 7 + 9;
            b = a + 10;
            c = b - a;

            return 0;
        }
    \end{minted}

\end{frame}

%%%%%%%%%%%%%%%%%%%%%%%%%%%%%%%%%%%%%%%%%%%%%%%%%%%%%%%%%%%%%%%%%%%%%%%%%%%%%%%
%%%%%%%%%%%%%%%%%%%%%%%%%%%%%%%%%%%%%%%%%%%%%%%%%%%%%%%%%%%%%%%%%%%%%%%%%%%%%%%
%%%%%%%%%%%%%%%%%%%%%%%%%%%%%%%%%%%%%%%%%%%%%%%%%%%%%%%%%%%%%%%%%%%%%%%%%%%%%%%
%%%%%%%%%%%%%%%%%%%%%%%%%%%%%%%%%%%%%%%%%%%%%%%%%%%%%%%%%%%%%%%%%%%%%%%%%%%%%%%
%%%%%%%%%%%%%%%%%%%%%%%%%%%%%%%%%%%%%%%%%%%%%%%%%%%%%%%%%%%%%%%%%%%%%%%%%%%%%%%
%%%%%%%%%%%%%%%%%%%%%%%%%%%%%%%%%%%%%%%%%%%%%%%%%%%%%%%%%%%%%%%%%%%%%%%%%%%%%%%

\section{Exercícios}

%%%%%%%%%%%%%%%%%%%%%%%%%%%%%%%%%%%%%%%%%%%%%%%%%%%%%%%%%
\begin{frame}[fragile]{Exercício 1}

    Qual o valor armazenado na variável \texttt{a} no fim do programa?

    \begin{minted}{c}
        int main() {
            int a, b, c, d;

            d = 3;
            c = 2;
            b = 4;
            d = c + b;
            a = d + 1;
            a = a + 1;

            return 0;
        }
    \end{minted}

\end{frame}

%%%%%%%%%%%%%%%%%%%%%%%%%%%%%%%%%%%%%%%%%%%
\begin{frame}[fragile]{Exercício 2}

    Qual erro existe neste programa?

    \begin{minted}{c}
        int main () {
            int a, b;
            double c, d;
            int g;

            d = 3.0;
            c = 2.4142;
            b = 4;
            a = 5 * b;
            d = b + 90;
            e = c * d;
            a = a + 1;

            return 0;
        }
    \end{minted}
\end{frame}


%%%%%%%%%%%%%%%%%%%%%%%%%%%%%%%%%%%%%%%%%
%%%%%%%%%%%%%%%%%%%%%%%%%%%%%%%%%%%%%%%%%
%%%%%%%%%%%%%%%%%%%%%%%%%%%%%%%%%%%%%%%%%
%%%%%%%%%%%%%%%%%%%%%%%%%%%%%%%%%%%%%%%%%
%%%%%%%%%%%%%%%%%%%%%%%%%%%%%%%%%%%%%%%%%
%%%%%%%%%%%%%%%%%%%%%%%%%%%%%%%%%%%%%%%%%
\section{Expressões e operadores aritméticos}

%%%%%%%%%%%%%%%%%%%%%%%%%%%%%%%%%%%%%%%%%
\begin{frame}[fragile]{Expressões}

    \begin{itemize}[<+->]
        \item Vimos que constantes e variáveis são expressões
        \item Uma expressão também pode ser um conjunto de operações aritméticas, lógicas ou relacionais utilizadas para fazer ``cálculos'' sobre os valores das variáveis.
    \end{itemize}

    \pause
    Exemplo de expressão:

    \begin{minted}{c}
        a + b
    \end{minted}
    Calcula a soma de {\bf a} e {\bf b}.

\end{frame}

%%%%%%%%%%%%%%%%%%%%%%%%%%%%%%%%%%%%%%%%%%%%%%%%%%%%%%%%%%%%%%%%%%%%%%%%

\begin{frame}[fragile]{Expressões aritméticas}

    \begin{itemize}[<+->]
        \item Os operadores aritméticos são: \texttt{+}, \texttt{-}, \texttt{*}, \texttt{/}, \texttt{\%}
        \item {\it expressão}  + {\it expressão} : Calcula a soma de duas expressões.

        Ex: {\tt 10 + 15;}

        \item {\it expressão}  - {\it expressão} : Calcula a subtração de duas expressões.

        Ex: {\tt 5 - 7;}

        \item {\it expressão}  * {\it expressão} : Calcula o produto de duas expressões.

        Ex: {\tt 3 * 4;}
    \end{itemize}

\end{frame}

%%%%%%%%%%%%%%%%%%%%%%%%%%%%%%%%%%%%%%%%%
\begin{frame}[fragile]{Expressões aritméticas}

    \begin{itemize}[<+->]
        \item {\it expressão}  / {\it expressão} : Calcula a divisão de duas expressões.

        Ex: {\tt 4 / 2;}

        \item {\it expressão}  \% {\it expressão} : Calcula o resto da divisão (inteira) de duas expressões.

        Ex: {\tt 5 \% 2;}

        \item - {\it expressão} : Inverte o sinal da expressão.

        Ex: {\tt -5;}
    \end{itemize}

\end{frame}


%%%%%%%%%%%%%%%%%%%%%%%%%%%%%%%%%%%%%%%%%
\begin{frame}[fragile]{Expressões aritméticas}

    Mais sobre o operador resto da divisão \%:
    \begin{itemize}[<+->]
        \item Quando computamos ``$a$ dividido por $b$'', isto tem como resultado um valor $p$ e um resto $r < b$ que são únicos tais que $$a = p*b +r$$
        \item Ou seja, $a$ pode ser dividido em $p$ partes inteiras de tamanho $b$, e sobrará um resto $r < b$.
    \end{itemize}

    \pause
    Exemplos:

    $5\%2$ tem como resultado o valor 1.\\
    $15\%3$ tem como resultado o valor 0.\\
    $1\%5$ tem como resultado o valor 1.\\
    $19\%4$ tem como resultado o valor 3.
\end{frame}

%%%%%%%%%%%%%%%%%%%%%%%%%%%%%%%%%%%%%%%%%

\begin{frame}[fragile]{Expressões aritméticas}

    No exemplo abaixo, quais valores serão impressos?

    \begin{minted}{c}
    #include <stdio.h>

    int main() {
        printf("%d\n", 29 % 3);
        printf("%d\n", 4 % 15);

        return 0;
    }
    \end{minted}

\end{frame}


%%%%%%%%%%%%%%%%%%%%%%%%%%%%%%%%%%%%%%%%%

\begin{frame}[fragile]{Expressões aritméticas}

    Mais sobre o operador de divisão /:
    \begin{itemize}[<+->]
        \item Quando utilizado sobre valores inteiros, o resultado da operação de divisão será inteiro.
        Isto significa que a parte fracionária da divisão será desconsiderada.
        \begin{itemize}
            \item $5/2$ tem como resultado o valor $2$.
        \end{itemize}
        \item Quando pelo menos um dos operandos for ponto flutuante, então a divisão será fracionária.
        Ou seja, o resultado será a divisão exata dos valores.
        \begin{itemize}
            \item $5.0 / 2$ tem como resultado o valor $2.5$.
        \end{itemize}
    \end{itemize}
\end{frame}

%%%%%%%%%%%%%%%%%%%%%%%%%%%%%%%%%%%%%%%%%

\begin{frame}[fragile]{Expressões aritméticas}

    No exemplo abaixo, quais valores serão impressos?

    \begin{minted}{c}
    #include <stdio.h>

    int main() {
        int a = 5, b = 2;
        float c = 5.0, d = 2.0;

        printf("%d\n",a/b);
        printf("%f\n", a/d);
        printf("%f\n", c/d);

        return 0;
    }
    \end{minted}

\end{frame}

%%%%%%%%%%%%%%%%%%%%%%%%%%%%%%%%%%%%%%%%%
\begin{frame}[fragile]{Expressões aritméticas}

    \begin{itemize}[<+->]
        \item As expressões aritméticas (e todas as expressões) operam sobre outras expressões.
        \item É possível compor expressões complexas como por exemplo:

        \texttt{a = b * ((2/c) + (9 + d*8));}

        \item Qual o valor da expressão \texttt{5 + 10 \% 3}?
        \item E da expressão \texttt{5 * 10 \% 3}?
    \end{itemize}

\end{frame}

%%%%%%%%%%%%%%%%%%%%%%%%%%%%%%%%%%%%%%%%%

\begin{frame}[fragile]{Precedência}
    \begin{itemize}[<+->]
        \item Precedência é a ordem na qual os operadores serão avaliados quando o programa for executado.
        Em C, os operadores são avaliados na seguinte ordem:

        \begin{itemize}
            \item \texttt{*} e \texttt{/}, na ordem em que aparecerem na expressão.
            \item \texttt{\%}
            \item \texttt{+} e \texttt{-}, na ordem em que aparecerem na expressão.
        \end{itemize}

        \item Exemplo: \texttt{8 + 10*6} é igual a 68.
    \end{itemize}
\end{frame}

%%%%%%%%%%%%%%%%%%%%%%%%%%%%%%%%%%%%%%%%%

\begin{frame}[fragile]{Alterando a precedência}
    \begin{itemize}[<+->]
        \item ({\it expressão}) também é uma expressão, que calcula o resultado da expressão dentro dos parênteses primeiro.
        \begin{itemize}
            \item 5 + 10 \% 3 é igual a 6
            \item (5 + 10) \% 3 é igual a 0
        \end{itemize}

        \item Você pode usar quantos parênteses desejar dentro de uma expressão.
        \item Use sempre parênteses em expressões para deixar claro em qual ordem a expressão é avaliada!
    \end{itemize}
\end{frame}

%%%%%%%%%%%%%%%%%%%%%%%%%%%%%%%%%%%%%%%%%
%%%%%%%%%%%%%%%%%%%%%%%%%%%%%%%%%%%%%%%%%
%%%%%%%%%%%%%%%%%%%%%%%%%%%%%%%%%%%%%%%%%
%%%%%%%%%%%%%%%%%%%%%%%%%%%%%%%%%%%%%%%%%
%%%%%%%%%%%%%%%%%%%%%%%%%%%%%%%%%%%%%%%%%
%%%%%%%%%%%%%%%%%%%%%%%%%%%%%%%%%%%%%%%%%
%%%%%%%%%%%%%%%%%%%%%%%%%%%%%%%%%%%%%%%%%

\section{Operadores $++$ e $--$}

%%%%%%%%%%%%%%%%%%%%%%%%%%%%%%%%%%%%%%%%%
\begin{frame}[fragile]{Incremento($++$) e decremento($--$)}

    \begin{itemize}[<+->]
        \item É muito comum escrevermos expressões para incrementar/decrementar o valor de uma variável por 1.
        \begin{minted}{c}
            a = a + 1;
            a = a - 1;
        \end{minted}

        \item Em C, o operador unário $++$ é  usado para incrementar de 1 o valor de uma variável.
        \begin{minted}{c}
            a++;
        \end{minted}

        \item O operador unário $--$ é usado para decrementar de 1 o valor de uma variável.
        \begin{minted}{c}
            a--;
        \end{minted}

    \end{itemize}
\end{frame}



%%%%%%%%%%%%%%%%%%%%%%%%%%%%%%%%%%%%%%%%%%
%%%%%%%%%%%%%%%%%%%%%%%%%%%%%%%%%%%%%%%%%
%%%%%%%%%%%%%%%%%%%%%%%%%%%%%%%%%%%%%%%%%%
%%%%%%%%%%%%%%%%%%%%%%%%%%%%%%%%%%%%%%%%%%
%%%%%%%%%%%%%%%%%%%%%%%%%%%%%%%%%%%%%%%%%
%%%%%%%%%%%%%%%%%%%%%%%%%%%%%%%%%%%%%%%%%
\section{Exercícios}

%%%%%%%%%%%%%%%%%%%%%%%%%%%%%%%%%%%%%%%%%
\begin{frame}[fragile]{Exercício}
    Crie um programa que computa e imprime a soma, a diferença, a multiplicação e divisão dos dois números do tipo double.
\end{frame}


%%%%%%%%%%%%%%%%%%%%%%%%%%%%%%%%%%%%%%%%%%
%%%%%%%%%%%%%%%%%%%%%%%%%%%%%%%%%%%%%%%%%
%%%%%%%%%%%%%%%%%%%%%%%%%%%%%%%%%%%%%%%%%%
%%%%%%%%%%%%%%%%%%%%%%%%%%%%%%%%%%%%%%%%%%
%%%%%%%%%%%%%%%%%%%%%%%%%%%%%%%%%%%%%%%%%
%%%%%%%%%%%%%%%%%%%%%%%%%%%%%%%%%%%%%%%%%


\section{Informações extras}

%%%%%%%%%%%%%%%%%%%%%%%%%%%%%%%%%%%%%%%%%

\begin{frame}[fragile]{Informações extras: incremento($++$) e decremento($--$)}
    Há uma diferença quando estes operadores são usados à esquerda ou à direita de uma variável \textbf{e} fazem parte de uma expressão maior:
    \pause
    \begin{itemize}[<+->]
        \item {\tt ++a}: Neste caso o valor de {\tt a} será incrementado antes e só depois o valor de {\tt a} é usado na expressão.
        \item {\tt a++}: Neste caso o valor de {\tt a} é usado na expressão, e só depois é incrementado.
        \item A mesma coisa acontece com o operador $--$.
    \end{itemize}
\end{frame}

%%%%%%%%%%%%%%%%%%%%%%%%%%%%%%%%%%%%%%%%%
\begin{frame}[fragile]{Informações extras: incremento($++$) e decremento($--$)}

    O que o programa abaixo imprime?

    \begin{minted}{c}
        #include <stdio.h>

        int main() {
            int a = 5, b;

            b = ++a;
            printf("b: %d\n", b);
            printf("a: %d\n", a);

            b = a++;
            printf("b: %d\n", b);
            printf("a: %d\n", a);

            return 0;
        }
    \end{minted}
\end{frame}

%%%%%%%%%%%%%%%%%%%%%%%%%%%%%%%%%%%%%%%%%%
\begin{frame}[fragile]{Informações extras: atribuições simplificadas}

    Uma expressão da forma

    \begin{minted}{c}
        a = a + b;
    \end{minted}

    Pode ser simplificada como

    \begin{minted}{c}
        a += b;
    \end{minted}

    A variável à qual o valor é atribuído faz parte da expressão.
\end{frame}


%%%%%%%%%%%%%%%%%%%%%%%%%%%%%%%%%%%%%%%%%
\begin{frame}[fragile]{Informações extras: atribuições simplificadas}

    \begin{center}
    \begin{tabular}{lll}
        \hline
        \textbf{Comando} & \textbf{Exemplo} & \textbf{Corresponde a:} \\
        \hline
        += & a += b; & a = a + b;  \\
        -= & a -= b; & a = a - b; \\
        *= & a *= b; & a = a * b; \\
        /= & a /= b; & a = a / b; \\
        \%= & a \%= b; & a = a \% b; \\
        \hline
    \end{tabular}
    \end{center}

\end{frame}

%%%%%%%%%%%%%%%%%%%%%%%%%%%%%%%%%%%%%%%%%


\end{document}
