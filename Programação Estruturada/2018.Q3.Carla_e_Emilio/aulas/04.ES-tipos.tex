% Copyright © 2015
% Eduardo Candido Xavier <eduardo@ic.unicamp.br>
%
% This work is free. You can redistribute it and/or modify it under the
% terms of the Do What The Fuck You Want To Public License, Version 2,
% as published by Sam Hocevar. See the COPYING file for more details.
%
% DO WHAT THE FUCK YOU WANT TO PUBLIC LICENSE
% Version 2, December 2004
%
% Copyright (C) 2004 Sam Hocevar <sam@hocevar.net>
%
% Everyone is permitted to copy and distribute verbatim or modified
% copies of this license document, and changing it is allowed as long
% as the name is changed.
%
% DO WHAT THE FUCK YOU WANT TO PUBLIC LICENSE
% TERMS AND CONDITIONS FOR COPYING, DISTRIBUTION AND MODIFICATION
%
% 0. You just DO WHAT THE FUCK YOU WANT TO.

\documentclass[handout]{beamer}
\usetheme{metropolis}
\beamertemplatetransparentcoveredhigh

\usepackage[portuges]{babel}
\usepackage{graphicx}
\graphicspath{{./figs/}}
\usepackage{listings}
\usepackage{color}
\usepackage{hyperref}
\usepackage{xpatch}
\usepackage[outputdir=build]{minted}

\makeatletter
\AtBeginEnvironment{minted}{\dontdofcolorbox}
\def\dontdofcolorbox{\renewcommand\fcolorbox[4][]{##4}}
\xpatchcmd{\inputminted}{\minted@fvset}{\minted@fvset\dontdofcolorbox}{}{}
\xpatchcmd{\mintinline}{\minted@fvset}{\minted@fvset\dontdofcolorbox}{}{}
\makeatother
\setminted[c]{
  linenos=true,
  breaklines=true,
  encoding=utf8,
  frame=lines,
  framerule=0.5pt,
  autogobble,
  fontsize=\small,
}
\setminted[bash]{
  linenos=true,
  encoding=utf8,
  frame=lines,
  framerule=0.5pt,
  autogobble,
  fontsize=\small
}

\newcommand{\cod}[1]{\mintinline{c}{#1}}


\definecolor{dkgreen}{rgb}{0,0.6,0}
\definecolor{gray}{rgb}{0.5,0.5,0.5}
\definecolor{mauve}{rgb}{0.58,0,0.82}


\definecolor{Purple}{HTML}{911146}
\definecolor{Orange}{HTML}{CF4A30}
\setbeamercolor{alerted text}{fg=Orange}
\setbeamercolor{frametitle}{bg=Purple}
\setbeamercolor{block body}{bg=Purple!20,fg=black}
\setbeamercolor{block title}{bg=Purple!50,fg=black}
\setbeamertemplate{blocks}[rounded][shadow=true]


\newcommand{\setcoverbg}{
  \setbeamertemplate{background}
  {\includegraphics[width=\paperwidth,height=\paperheight]{backgrounds/coverbg}}
}
\newcommand{\setsectionbg}{
  \setbeamertemplate{background}
  {\includegraphics[width=\paperwidth,height=\paperheight]{backgrounds/blank}}
}

\title{Programação Estruturada}
\subtitle{Entrada e Saída. Mais sobre dados}

\author{Professores Emílio Francesquini e Carla Negri Lintzmayer}
\institute{Centro de Matemática, Computação e Cognição\\ Universidade Federal do ABC}
\date{2018.Q3}

\begin{document}

\setcoverbg
\maketitle
\setsectionbg



%%%%%%%%%%%%%%%%%%%%%%%%%%%%%%%%%%%%%%%%%
%%%%%%%%%%%%%%%%%%%%%%%%%%%%%%%%%%%%%%%%%
%%%%%%%%%%%%%%%%%%%%%%%%%%%%%%%%%%%%%%%%%
%%%%%%%%%%%%%%%%%%%%%%%%%%%%%%%%%%%%%%%%%
%%%%%%%%%%%%%%%%%%%%%%%%%%%%%%%%%%%%%%%%%
%%%%%%%%%%%%%%%%%%%%%%%%%%%%%%%%%%%%%%%%%
\section{Saída de dados: printf}

%%%%%%%%%%%%%%%%%%%%%%%%%%%%%%%%%%%%%%%%%
\begin{frame}[fragile]{Escrevendo na tela}
    Para imprimir um texto, utilizamos o comando \cod{printf}.

    O texto deve ser uma constante do tipo \textbf{string}.

    \begin{minted}{c}
        printf("Olá Pessoal!");
    \end{minted}
    
    Saída:
    \begin{minted}{bash}
        Olá Pessoal!
    \end{minted}
\end{frame}

%%%%%%%%%%%%%%%%%%%%%%%%%%%%%%%%%%%%%%%%%
\begin{frame}[fragile]{Escrevendo na tela}
    No meio da constante {\bf string} pode-se incluir caracteres de formatação especiais.

    O símbolo especial \cod{\n} é responsável por pular uma linha na saída.

    \begin{minted}{c}
        printf("Olá Pessoal!\nOlá Pessoal");
    \end{minted}
    
    Saída:
    \begin{minted}{bash}
        Olá Pessoal!
        Olá Pessoal
    \end{minted}
\end{frame}

%%%%%%%%%%%%%%%%%%%%%%%%%%%%%%%%%%%%%%%%%
\begin{frame}[fragile]{Escrevendo o conteúdo de uma variável na tela}
    Podemos imprimir também o conteúdo de variáveis.
    \pause
    
    Para isso utilizamos símbolos especiais no texto, para representar que aquele trecho deve ser substituído por uma variável ou constante.
    \pause
    
    No final do \cod{printf} passamos uma lista de variáveis ou constantes, separadas por vírgula.
    \pause

    \begin{minted}{c}
        int a = 10;
        printf("A variável %s contém o valor %d", "a", a);
    \end{minted}
    
    Saída:
    \begin{minted}{bash}
        A variável a contém o valor 10
    \end{minted}
\end{frame}

%%%%%%%%%%%%%%%%%%%%%%%%%%%%%%%%%%%%%%%%%
\begin{frame}[fragile]{Escrevendo o conteúdo de uma variável na tela}
    No comando
    \begin{minted}{c}
        int a = 10;
        printf("A variável %s contém o valor %d", "a", a);
    \end{minted}
    \texttt{\%s} vai ser substituído por uma variável/constante do tipo \cod{string} e \texttt{\%d} vai ser substituído por uma variável/constante do tipo \cod{int}.

    \pause
    É importante que a ordem na lista final de variáveis ou constantes seja a mesma ordem que os símbolos aparecem no texto.

    O comando a seguir está errado:
    \begin{minted}{c}
        int a = 10;
        printf("A variável %s contém o valor %d", a, "a");
    \end{minted}

\end{frame}

%%%%%%%%%%%%%%%%%%%%%%%%%%%%%%%%%%%%%%%%%
\begin{frame}[fragile]{Formatos inteiros}

    \textbf{\%d} --- Escreve um inteiro na tela.

    \begin{minted}{c}
        printf("%d\n", 10);
    \end{minted}
    Saída:
    \begin{minted}{bash}
        10
    \end{minted}

    \pause
    \begin{minted}{c}
        int a = 12;
        printf("O valor é %d\n", a);
    \end{minted}
    Saída:
    \begin{minted}{bash}
        O valor é 12
    \end{minted}
\end{frame}

%%%%%%%%%%%%%%%%%%%%%%%%%%%%%%%%%%%%%%%%%
\begin{frame}[fragile]{Formatos inteiros}

    O formato {\bf \%d} pode ser substituído por {\bf \%u} e {\bf \%ld} se desejarmos escrever variáveis \cod{unsigned} ou \cod{long int}, respectivamente.

    \begin{minted}{c}
        printf("%d\n", 4000000000);
    \end{minted}
    Saída:
    \begin{minted}{bash}
        -294967296
    \end{minted}

    \pause
    \begin{minted}{c}
        printf("%ld\n", 4000000000);
    \end{minted}
    Saída:
    \begin{minted}{bash}
        4000000000
    \end{minted}
\end{frame}

%%%%%%%%%%%%%%%%%%%%%%%%%%%%%%%%%%%%%%%%%
\begin{frame}[fragile]{Formatos ponto flutuante}

    \textbf{\%f} --- Escreve um ponto flutuante na tela.

    \begin{minted}{c}
        printf("%f\n", 10.0);
    \end{minted}
    Saída:
    \begin{minted}{bash}
        10.000000
    \end{minted}

\end{frame}

%%%%%%%%%%%%%%%%%%%%%%%%%%%%%%%%%%%%%%%%%
\begin{frame}[fragile]{Formatos ponto flutuante}

    \textbf{\%.$N$f} --- Escreve um ponto flutuante na tela com $N$ casas decimais.

    \begin{minted}{c}
        printf("%.2f\n", 10.1111);
    \end{minted}
    Saída:
    \begin{minted}{bash}
        10.11
    \end{minted}

\end{frame}

%%%%%%%%%%%%%%%%%%%%%%%%%%%%%%%%%%%%%%%%%
\begin{frame}[fragile]{Formatos ponto flutuante}

    O formato {\bf \%f} pode ser substituído por {\bf \%lf} para escrever um \cod{double} ao invés de um \cod{float}.

    \begin{minted}{c}
        double a = 10.0;
        printf("%.2lf\n", a);
    \end{minted}
    Saída:
    \begin{minted}{bash}
        10.00
    \end{minted}

\end{frame}

%%%%%%%%%%%%%%%%%%%%%%%%%%%%%%%%%%%%%%%%%
\begin{frame}[fragile]{Formato caractere}

    \textbf{\%c} --- Escreve um caractere.

    \begin{minted}{c}
        printf("%c\n", 'A');
    \end{minted}
    Saída:
    \begin{minted}{bash}
        A
    \end{minted}

    Veremos que o comando \texttt{printf("\%c\textbackslash n", 65)} também imprime a letra A.
\end{frame}

%%%%%%%%%%%%%%%%%%%%%%%%%%%%%%%%%%%%%%%%%
\begin{frame}[fragile]{Formato {\bf string}}

    \textbf{\%s} --- Escreve uma {\bf string}.

    \begin{minted}{c}
        printf ("%s\n", "Meu primeiro programa");
    \end{minted}
    Saída:
    \begin{minted}{bash}
        Meu primeiro programa
    \end{minted}
\end{frame}


%%%%%%%%%%%%%%%%%%%%%%%%%%%%%%%%%%%%%%%%%
%%%%%%%%%%%%%%%%%%%%%%%%%%%%%%%%%%%%%%%%%
%%%%%%%%%%%%%%%%%%%%%%%%%%%%%%%%%%%%%%%%%
%%%%%%%%%%%%%%%%%%%%%%%%%%%%%%%%%%%%%%%%%
%%%%%%%%%%%%%%%%%%%%%%%%%%%%%%%%%%%%%%%%%
%%%%%%%%%%%%%%%%%%%%%%%%%%%%%%%%%%%%%%%%%

\section{Entrada de dados: scanf}

%%%%%%%%%%%%%%%%%%%%%%%%%%%%%%%%%%%%%%%%%
\begin{frame}[fragile]{A função {\bf scanf}}

    \begin{itemize}[<+->]
        \item Realiza a leitura de dados a partir do teclado.
        \item Parâmetros:
        \begin{itemize}
            \item Uma {\bf string}, indicando os tipos das variáveis que serão lidas e o formato dessa leitura.
            \item Uma lista de variáveis.
        \end{itemize}
        \item Aguarda que o usuário digite um valor e atribui o valor digitado à variável.
    \end{itemize}
\end{frame}

%%%%%%%%%%%%%%%%%%%%%%%%%%%%%%%%%%%%%%%%%
\begin{frame}[fragile]{A função {\bf scanf}}

    O programa abaixo é composto de quatro passos:
    \begin{enumerate}
        \item Cria uma variável {\bf n}
        \item Escreve na tela a string ``Digite um número:''
        \item Lê o valor do número digitado e o salva em \textbf{n}
        \item Imprime o valor do número digitado
    \end{enumerate}

    \begin{minted}{c}
        #include <stdio.h>
        int main() {
            int n;
            printf("Digite um número: ");
            scanf("%d", &n); /* note a presença do caractere & */
            printf("O valor digitado foi %d\n",n);
            return 0;
        }
    \end{minted}
\end{frame}

%%%%%%%%%%%%%%%%%%%%%%%%%%%%%%%%%%%%%%%%%
\begin{frame}[fragile]{Formatos de leitura de variável}
    No \cod{scanf}, cada variável para onde será lido um valor deve ser precedida do caractere {\bf \&}.
    \pause

    Os formatos de leitura são muito semelhantes aos formatos de escrita do {\bf printf}.
  
    \begin{center}
        \begin{tabular} {|l|l|}
            \hline \textbf{Código} & \textbf{Função} \\
            \hline  \%c & Lê um único caractere \\
            \%s & Lê uma série de caracteres (string) \\
            \%d & Lê um número decimal \\
            \%u & Lê um decimal sem sinal \\
            \%ld & Lê um inteiro longo \\
            \%f & Lê um número em ponto flutuante \\
            \%lf & Lê um double \\ \hline
        \end{tabular}
    \end{center}
\end{frame}

%%%%%%%%%%%%%%%%%%%%%%%%%%%%%%%%%%%%%%%%%
\begin{frame}[fragile]{A função {\bf scanf}}
    \begin{minted}[fontsize=\scriptsize]{c}
        #include <stdio.h>

        int main() {
            char c;
            float b;
            int a;

            printf("Digite um caractere: ");
            scanf("%c", &c);
            printf("Digite um ponto flutuante: ");
            scanf("%f", &b);
            printf("Digite um inteiro: ");
            scanf("%d", &a);

            printf("Os dados lidos foram: %c, %f e %d.\n",c,b,a);

            return 0;
        }
    \end{minted}
\end{frame}

%%%%%%%%%%%%%%%%%%%%%%%%%%%%%%%%%%%%%%%%%
\begin{frame}[fragile]{A função {\bf scanf}}
    No exemplo anterior, os textos ``Digite \ldots'' aparecem na saída padrão.

    \begin{minted}[fontsize=\scriptsize]{c}
        #include <stdio.h>

        int main() {
            char c;
            float b;
            int a;

            scanf("%c", &c);
            scanf("%f", &b);
            scanf("%d", &a);

            printf("Os dados lidos foram: %c, %f e %d.\n",c,b,a);

            return 0;
        }
    \end{minted}
\end{frame}

%%%%%%%%%%%%%%%%%%%%%%%%%%%%%%%%%%%%%%%%%
\begin{frame}[fragile]{A função {\bf scanf}}

    É possível ler várias variáveis em um mesmo comando \cod{scanf}, basta especificar todos os formatos das variáveis a serem lidas e depois as variáveis separadas por vírgulas.

    \begin{minted}{c}
        #include <stdio.h>

        int main() {
            int m, n, o;
            printf("Digite três números: ");
            scanf("%d %d %d", &m, &n, &o);
            printf("O valores digitados foram %d, %d e %d\n", m, n, o);

            return 0;
        }
    \end{minted}
\end{frame}

%%%%%%%%%%%%%%%%%%%%%%%%%%%%%%%%%%%%%%%%%%%%%%%%%
%%%%%%%%%%%%%%%%%%%%%%%%%%%%%%%%%%%%%%%%%%%%%%%%%
%%%%%%%%%%%%%%%%%%%%%%%%%%%%%%%%%%%%%%%%%%%%%%%%%
%%%%%%%%%%%%%%%%%%%%%%%%%%%%%%%%%%%%%%%%%%%%%%%%%
%%%%%%%%%%%%%%%%%%%%%%%%%%%%%%%%%%%%%%%%%%%%%%%%%
%%%%%%%%%%%%%%%%%%%%%%%%%%%%%%%%%%%%%%%%%%%%%%%%%

\section{Informações extras sobre tipos}

%%%%%%%%%%%%%%%%%%%%%%%%%%%%%%%%%%%%%%%%%%%%%%%%%
\begin{frame}[fragile]{Constantes inteiras}

    Possíveis formas de escrita em C de um número inteiro:

    \begin{itemize}[<+->]
        \item Normalmente

        Ex: \cod{10}, \cod{145}, \cod{1000000}

        \item Na forma hexadecimal (base 16), precedido de 0x

        Ex: \cod{0xA} ($0xA_{16} = 10$), \cod{0x100} ($0x100_{16} = 256$)

        \item Na forma octal (base 8), precedido de 0

        Ex: \cod{010} ($0x10_{8} = 8$)
    \end{itemize}
\end{frame}

%%%%%%%%%%%%%%%%%%%%%%%%%%%%%%%%%%%%%%%%%%%%%%%%%%
\begin{frame}[fragile]{Constantes do tipo ponto flutuante}

    \begin{itemize}[<+->]
      \item Um número só pode ser considerado ponto flutuante se tiver uma parte ``não inteira'', mesmo que ela tenha valor zero.

      Ex: \cod{10.0}, \cod{5.0}, \cod{4.7}

      \item Um número ponto flutuante pode ser escrito em notação científica: um número seguido da letra {\bf e} e um expoente. Um número escrito dessa forma deve ser interpretado como: $$\mbox{numero} \times 10^{\mbox{expoente}}$$

      Ex: \cod{2e2} ($2 \cdot 10^2 = 200.0$)
    \end{itemize}
\end{frame}

%%%%%%%%%%%%%%%%%%%%%%%%%%%%%%%%%%%%%%%%%%%%%%%
\begin{frame}[fragile]{Caracteres}
    \begin{itemize}[<+->]
        \item Caracteres são, na verdade, variáveis inteiras que armazenam um número associado ao símbolo
        
        \item Talvez a tabela de símbolos mais famosa utilizada pelos computadores é a tabela ASCII\footnote{Pronúncia correta: ás-ki.} (\emph{American Standard Code for Information Interchange}), mas existem outras (EBCDIC, Unicode, etc.)
        
        \item Uma variável do tipo {\tt char} armazena um símbolo (no caso, o inteiro correspondente ao símbolo). Seu valor pode ir de -128 a 127 (1 byte).

    % \item {\tt unsigned char:} Armazena um símbolo (no caso, o inteiro correspondente).  Seu valor pode ir de 0 a 255.
  \end{itemize}
\end{frame}

%%%%%%%%%%%%%%%%%%%%%%%%%%%%%%%%%%%%%%%%%%%%%%%%
\begin{frame}{Tabela ASCII}
    \scriptsize
    \begin{tabular}{r|c|c|c|c|c|c|c|c|c|c|c|c|c|c|c|c|}
        & 0 & 1 & 2 & 3 & 4 & 5 & 6 & 7 & 8 & 9 &10 &11 &12 &13 &14 &15 \\
        \hline
        0 & \multicolumn{16}{|c|}{caracteres de Controle}\\
        16 & \multicolumn{16}{|c|}{}\\
        \hline
        32 &   & ! & " & \# & \$ & \% & \& & ' & ( & ) & * & + & , & - & . & / \\
        \hline
        48 & 0 & 1 & 2 & 3 & 4 & 5 & 6 & 7 & 8 & 9 & : & ; & $<$ & = & $>$ & ? \\
        \hline
        64 & @ & A & B & C & D & E & F & G & H & I & J & K & L & M & N & O \\
        \hline
        80 & P & Q & R & S & T & U & V & W & X & Y & Z & $[$ & $\backslash$ & $]$ & $\wedge$ & \_ \\
        \hline
        96 & \`{} & a & b & c & d & e & f & g & h & i & j & k & l & m & n & o \\
        \hline
        112 & p & q & r & s & t & u & v & w & x & y & z & \{ & $\vert$ & \} & $\sim$ &  \\
        \hline
    \end{tabular}
\end{frame}

%%%%%%%%%%%%%%%%%%%%%%%%%%%%%%%%%%%%%%%%%%%%%%%%%%%%%
\begin{frame}[fragile]{Caracteres}
    Toda constante do tipo caractere pode ser usada como uma constante do tipo inteiro e vice-versa.
    
    Nesse caso, o valor atribuído será o valor daquela letra na tabela ASCII.

    \begin{minted}{c}
        int a;
        char b;

        b = 65;
        printf("Valor de b: %c\n", b);

        a = 'C';
        printf("Valor de a: %d\n", a);
    \end{minted}
\end{frame}

%%%%%%%%%%%%%%%%%%%%%%%%%%%%%%%%%%%%%%%%%%%%%%%%%%%%%%
\begin{frame}[fragile]{Caracteres}
    Qual a saída do programa a seguir?
    \begin{minted}{c}
        char b = 130;

        printf("%c %d\n", b, b);
    \end{minted}
\end{frame}


%%%%%%%%%%%%%%%%%%%%%%%%%%%%%%%%%%%%%%%%%%%%%%%%%%%%%%%%%%%%%%%
%%%%%%%%%%%%%%%%%%%%%%%%%%%%%%%%%%%%%%%%%%%%%%%%%%%%%%%%%%%%%%%
%%%%%%%%%%%%%%%%%%%%%%%%%%%%%%%%%%%%%%%%%%%%%%%%%%%%%%%%%%%%%%%
%%%%%%%%%%%%%%%%%%%%%%%%%%%%%%%%%%%%%%%%%%%%%%%%%%%%%%%%%%%%%%%
%%%%%%%%%%%%%%%%%%%%%%%%%%%%%%%%%%%%%%%%%%%%%%%%%%%%%%%%%%%%%%%
%%%%%%%%%%%%%%%%%%%%%%%%%%%%%%%%%%%%%%%%%%%%%%%%%%%%%%%%%%%%%%%

\section{Comentários}

%%%%%%%%%%%%%%%%%%%%%%%%%%%%%%%%%%%%%%%%%%%%%%%%%%%%%%%%%%
\begin{frame}[fragile]{Comentários}
    \begin{itemize}
        \item É comum incluir comentários em um programa para explicar certas partes do código e ajudar na sua compreensão.
        \item Os comentários são ignorados pelo compilador.
        \item Um comentário de uma linha deve ser escrito depois de {\bf //}\footnote{Comentários de uma linha não são permitidos em ANSI C.}.
        \item Um comentário também pode ser escrito entre \textbf{/*} e \textbf{*/}. Neste caso pode haver mais de uma linha de comentário.
  \end{itemize}
\end{frame}

%%%%%%%%%%%%%%%%%%%%%%%%%%%%%%%%%%%%%%%%%%%%%%%%%%%%%%%%%%
\begin{frame}[fragile]{Comentários}
    \begin{minted}{c}
        #include <stdio.h>

        int main() {
            int a; //isto é um comentário

            int b; /* isto
                      também é
                      um comentário */

            a = 2;
            b = 3;
            a = a + b;

            return 0;
        }
    \end{minted}
\end{frame}

%%%%%%%%%%%%%%%%%%%%%%%%%%%%%%%%%%%%%%%%%
%%%%%%%%%%%%%%%%%%%%%%%%%%%%%%%%%%%%%%%%%
%%%%%%%%%%%%%%%%%%%%%%%%%%%%%%%%%%%%%%%%%
%%%%%%%%%%%%%%%%%%%%%%%%%%%%%%%%%%%%%%%%%
%%%%%%%%%%%%%%%%%%%%%%%%%%%%%%%%%%%%%%%%%
%%%%%%%%%%%%%%%%%%%%%%%%%%%%%%%%%%%%%%%%%
\section{Conversão de tipos}

%%%%%%%%%%%%%%%%%%%%%%%%%%%%%%%%%%%%%%%%%
\begin{frame}[fragile]{Conversão de tipos}
    \begin{itemize}[<+->]
        \item É possível converter alguns tipos entre si, o que pode ser da forma implícita ou explícita.
        \item Na forma \textbf{implícita}, atribui-se diretamente um dado de um tipo para uma variável de outro tipo.
        \item Para isso, a capacidade (tamanho) do destino deve ser maior ou igual ao tamanho da origem, senão há perda de informação.
    \end{itemize}
\end{frame}

%%%%%%%%%%%%%%%%%%%%%%%%%%%%%%%%%%%%%%%%%
\begin{frame}[fragile]{Conversão implícita de tipos}
    \begin{minted}[fontsize=\scriptsize]{c}
        #include <stdio.h>
        int main() {
            int a = 9;
            double b;

            b = a;
            printf("a: %d e b: %lf\n", a, b);
            
            b = 5.56;
            a = b;
            printf("a: %d e b: %lf\n", a, b);
            
            b = 4000000000.56;
            a = b;
            printf("a: %d e b: %lf\n", a, b);
            
            printf("Tamanho em bytes de um double: %ld\n", sizeof(double));
            printf("Tamanho em bytes de um int: %ld\n", sizeof(int));
            return 0;
        }
      \end{minted}
\end{frame}

%%%%%%%%%%%%%%%%%%%%%%%%%%%%%%%%%%%%%%%%%
\begin{frame}[fragile]{Conversão de tipos}

    Na conversão \textbf{explícita}, nós explicitamente informamos o tipo para o qual o valor da variável ou expressão será convertido.

    \begin{minted}{c}
        #include <stdio.h>

        int main() {
            double b;

            b = ((double) 5 / (double) 2);
            printf("%lf\n", b);

            b =  5 / 2;
            printf("%lf\n", b);

            return 0;
        }
    \end{minted}
\end{frame}

%%%%%%%%%%%%%%%%%%%%%%%%%%%%%%%%%%%%%%%%%
\begin{frame}[fragile]{Um uso da conversão de tipos}

    A operação de divisão (\cod{/}) possui dois modos de operação de acordo com os seus argumentos: inteira ou de ponto flutuante.
    \begin{itemize}
        \item Se os dois argumentos forem inteiros, acontece a divisão inteira.
        A expressão \cod{10 / 3} tem como valor \cod{3}.
        \item Se \alert{um} dos dois argumentos for de ponto flutuante, acontece a divisão de ponto flutuante.
        A expressão \cod{1.5 / 3} tem como valor \cod{0.5}.
    \end{itemize}

    \pause
    Quando se deseja obter o valor de ponto flutuante de uma divisão (não-exata) de dois inteiros, basta converter um deles para ponto flutuante: a expressão \cod{10 / (float) 3} tem como valor \cod{3.33333333}.

    \pause
    Isso é especialmente útil quando se tem variáveis na expressão.
\end{frame}

%%%%%%%%%%%%%%%%%%%%%%%%%%%%%%%%%%%%%%%%%

\end{document}
