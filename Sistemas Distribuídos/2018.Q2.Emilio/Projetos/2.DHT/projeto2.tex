\documentclass[oneside,12pt,a4paper]{article}

\usepackage{polyglossia}
\setdefaultlanguage{brazil}
%\disablehyphenation

\usepackage{fontspec}
\setmainfont[Ligatures=TeX]{Linux Libertine O} % Calibri?
\setsansfont[Ligatures=TeX]{Linux Biolinum O}
\setmonofont[Ligatures=TeX]{Inconsolata}

\usepackage{amsmath}

\usepackage[xetex,hidelinks]{hyperref}

\usepackage{minted}

\title{Projeto 2\\ Tabela de Hash Distribuída (DHT)}
\author{Profs. Emilio Francesquini e Fernando Teubl Ferreira\\
  \texttt{\{e.francesquini,fernando.teubl\}@ufabc.edu.br} \\
  \mbox{}\\
  Centro de Matemática, Computação e Cognição\\
  Universidade Federal do ABC}

\date{Entrega: \textbf{27 de agosto de 2018}}

\begin{document}

\maketitle

\section{Descrição Geral}

Neste projeto vamos implementar uma tabela de hash distribuída (DHT)
que apesar da sua semelhança com o Chord é bem mais simples. A sua
implementação de DHT deverá ser capaz de manter a estrutura em anel e
dar suporte a:

\begin{itemize}
\item \textbf{Entrada e saída de nós da rede}. Você pode assumir que
  os nós nunca falham e que sempre avisam os outros nós sobre sua
  partida para que a estrutura da rede possa ser mantida.
\item \textbf{Inclusão, remoção e pesquisa de pares de chave-valor}.
\end{itemize}

Reforçando que \emph{não é necessário} implementar o protocolo
completo do Chord! A sua implementação deve ser capaz apenas de
garantir que os pontos descritos acima funcionam utilizando uma
topologia em anel.

Para verificar o funcionamento de sua implementação de DHT, cada
grupo deverá, também, implementar uma aplicação distribuída que a
utilize e demonstre o seu funcionamento.

\section{O projeto}

Este projeto compreende duas tarefas principais:

\begin{itemize}
\item Implementar uma DHT cuja topologia de conexão seja um anel
\item Implementar uma aplicação de exemplo que utilize a sua DHT
\end{itemize}

Em seguida detalhamos cada uma destas tarefas.

\subsection{DHT}\label{sec:dht}

Ao contrário do Chord que mantém uma \emph{finger table} para
diversos nós participantes da rede, cada nó participante da nossa DHT
só vai ter dois apontadores: um para o nó sucessor e outro para ou seu
antecessor. Isso quer dizer que nossa DHT não vai ter as mesmas
garantias relativas ao número de mensagens e nós afetados por
inserções e buscas que Chord oferece ($O(\log{}n)$).

Assim como no Chord, em nossa DHT:

\begin{itemize}
\item A cada nó é atribuído um identificador aleatório de m-bits
\item A cada entidade armazenada na DHT é atribuída uma chave de m-bits
\item Entidades com chave \texttt{k} estão sob responsabilidade do nó
  cujo identificador \texttt{id} seja o menor possível tal que
  $\text{id} \geq k$
\item Chamamos de nó \texttt{p} o nó cujo identificador é \texttt{p}
\end{itemize}

Para que a DHT seja utilizável pelas mais diversas aplicações é
necessário que ela tenha uma API bem definida. As seguintes operações
são \underline{sugestões} das operações mínimas que a API da sua DHT
deve oferecer:

\begin{itemize}
\item \texttt{join(KnownHostList)} Esta operação será utilizada pelas
  aplicações no momento que desejarem se conectar à DHT. Como
  parâmetro a operação recebe uma lista de nós conhecidos
  participantes da rede. Note que é uma lista de possíveis
  participantes já que eles podem não estar mais em execução. Neste
  caso a sua operação deve testar os elementos da lista até encontrar
  um participante ativo e se conectar a ele. Caso nenhum dos elementos
  da lista possa ser alcançado, a sua implementação deve assumir que é
  o primeiro nó da rede e iniciar uma nova DHT. Ao final desta chamada
  devem ter sido atualizados:
  \begin{itemize}
  \item Os campos \texttt{sucessor} e \texttt{predecessor} do nó ingressante
  \item O campo \texttt{predecessor} do sucessor do nó ingressante
  \item O campo \texttt{sucessor} do predecessor do nó ingressante
  \end{itemize}

  Também devem ter sido transferidos do sucessor do nó ingressante os
  dados armazenados na DHT que agora são responsabilidade do novo nó.

\item \texttt{leave()} Esta operação é chamada pela aplicação quando
  deseja se desconectar da rede. Ao final desta chamada, o anel da sua
  DHT deve ter sido atualizado para estar consistente, ou seja, os
  ponteiros dos vizinhos devem ter sido atualizados e os e os
  eventuais valores que estivessem sendo armazenados pelo nó que está
  partindo devem ter sido transferidos para o seu sucessor.

\item \texttt{store(key,value)} Armazena o valor \texttt{value} na DHT
  utilizando a chave \texttt{key}. Mais precisamente, armazena o valor
  \texttt{value} no nó cujo identificador seja
  \texttt{sucessor(hash(key))}. Note que a função de hash utilizada
  deve produzir um valor com m-bits, o mesmo número dos
  identificadores utilizados pelos nós.

\item \texttt{retrieve(key)} Efetua uma busca na DHT e devolve o valor
  \texttt{value} (caso presente na DHT) que havia sido previamente
  armazenado através de uma chamada à operação
  \texttt{store(key,value)}.

\end{itemize}

Você pode assumir que as chaves utilizadas nas funções \texttt{store}
e \texttt{retrieve} são simples strings. Já para os valores, você vai
ter que determinar que tipo de dados deverá ser armazenado para que a
sua aplicação de exemplo funcione corretamente. Para que sua API seja
a mais genérica possível, a sugestão é que o tipo dos valores seja um
vetor de bytes. Desta maneira qualquer tipo de objeto serializado
seria facilmente armazenado.


\subsection{Aplicação}

Você deverá implementar uma aplicação simples que demonstre o
funcionamento da sua DHT. Você pode escolher uma das sugestões abaixo
ou inventar a sua própria.

\begin{itemize}
\item Chat
\item Álbum de fotos
\item Sistema de arquivos distribuído
\item \ldots
\end{itemize}

\textbf{Importante:} \underline{Todas} as operações da sua API devem
ser utilizadas pela sua aplicação de exemplo.

\section{Implementação}

Você é livre para escolher a linguagem de programação que desejar para
este projeto. Isto só torna mais importante o fato de que o seu
relatório deverá trazer obrigatoriamente instruções detalhadas sobre
como compilar e executar a sua aplicação. Também tenha em mente que o
ambiente e utilizado para correção será uma máquina rodando Linux e
que deve ser possível utilizar sua aplicação neste SO. Sua nota será
\underline{drasticamente reduzida} caso o professor não consiga
compilar e executar o seu projeto utilizando as instruções fornecidas
no seu relatório.

Para implementar a API descrita acima será necessário definir um
protocolo de comunicação a ser seguido pelos nós participantes da
DHT. O protocolo abaixo é uma \underline{sugestão} de como fazê-lo.

\subsection{Protocolo}

Comunicações entre nós participantes da DHT podem ser feitas
utilizando \emph{sockets} TCP/IP. Se você desejar, poderá também
utilizar JRMI (ou o equivalente na linguagem de programação que você
escolher) para implementar a comunicação entre os nós. Cada um dos
métodos tem as suas vantagens, porém o uso de TCP/IP acaba deixando mais
claro o fluxo de mensagens entre os participantes da rede para o
programador.

Cada um dos nós participantes da rede poderá ter mais de uma conexão
aberta simultaneamente, por exemplo, uma com o sucessor e outra com o
predecessor. Quando há apenas um nó presente na DHT há zero conexão
aberta.

No protocolo que sugerimos, os nós farão a comunicação usando um
protocolo baseado em texto puro\footnote{ Uma implementação de DHT
  para uso real provavelmente privilegiaria um protocolo binário em
  vez de um protocolo baseado em texto puro. Neste projeto, por
  simplicidade, abrimos mão do possível ganho de desempenho que um
  protocolo binário propiciaria.}  que seguirá o seguinte formato:

\begin{table}[h]
  \begin{tabular}{|l|l|l|l|l|}
    \cline{1-1} \cline{3-3} \cline{5-5}
    \textbf{OP} &  & \textbf{Argumentos} &  & \textbf{\textbackslash{}n} \\ \cline{1-1} \cline{3-3} \cline{5-5}
  \end{tabular}
\end{table}

Onde \textbf{OP} identifica o tipo de mensagem e \textbf{Argumentos}
contém todos os argumentos necessários para que aquela mensagem possa
ser processada pelo destinatário. Caso o tipo da mensagem exija mais
de um argumento, eles são enviados um seguido do outro, todos na mesma
linha separados por espaços. Cada mensagem termina quando o caractere
de quebra de linha (\texttt{\textbackslash{}n}) for encontrado.


As seguintes mensagens são necessárias para o funcionamento do
protocolo:

\begin{itemize}
\item \textbf{JOIN} Recebe como argumentos o identificador, o endereço
  IP e a porta do nó ingressante. O identificador é um número de m
  bits sem sinal que pode ser enviado utilizando a sua representação
  em decimal. Já o IP é enviado no formato decimal separado por
  pontos. Por exemplo: ``127.0.0.1'' e a porta como um número inteiro
  decimal (ex. ``12345''). A mensagem de \texttt{JOIN} é roteada ao nó
  atualmente responsável pelo identificador do nó ingressando na rede
  que, segundo as regras definidas na Seção~\ref{sec:dht}, será o seu
  sucessor.

\item \textbf{JOIN\_OK} Esta mensagem é enviada a um nó ingressante em
  resposta a uma mensagem do tipo \texttt{JOIN} enviada por este.
  Como argumentos a mensagem \texttt{JOIN\_OK} tem o identificador do
  nó, o endereço IP e a porta do predecessor e sucessor a serem
  utilizados pelo novo nó. O sucessor é, na verdade, o próprio nó que
  está enviando a mensagem \texttt{JOIN\_OK} e o predecessor é o seu
  atual predecessor. A mensagem \texttt{JOIN\_OK} vai ser seguida
  imediatamente por 0 ou mais mensagens do tipo \texttt{TRANSFER} que
  conterão as chaves e valores que são responsabilidade do novo nó.

\item \textbf{NEW\_NODE} Quando um novo nó ingressa na DHT, a sua
  mensagem \texttt{JOIN} é enviada ao antigo responsável pelo
  identificador do novo nó, ou seja, seu sucessor. O no ingressante
  então recebe o seu predecessor via mensagem \texttt{JOIN\_OK} do seu
  sucessor. O nó ingressante deve, então, enviar uma mensagem
  \texttt{NEW\_NODE} ao seu predecessor notificando-o que ele deve
  atualizar o seu sucessor para apontar para o nó
  ingressante. \texttt{NEW\_NODE} recebe como parâmetros o endereço IP e
  porta do nó ingressante.

\item \textbf{LEAVE} Esta mensagem é enviada por um nó ao seu sucessor
  quando ele deseja deixar a DHT. Os argumentos para esta mensagem são
  o identificador, endereço IP e porta do predecessor do nó que está
  partindo. Uma mensagem \texttt{LEAVE} é seguida imediatamente de
  zero ou mais mensagens \texttt{TRANSFER} que se encarregam de
  transferir os dados armazenados no nó que está partindo para o seu
  sucessor.

\item \textbf{NODE\_GONE} Quando um nó deixa a rede, ele envia a
  mensagem \texttt{LEAVE} ao seu sucessor. O nó que está partindo
  também avisa ao seu predecessor que ele deve atualizar o seu
  sucessor para que aponte para o sucessor do nó que está saindo. Esta
  mensagem recebe 3 argumentos: o identificador, IP e porta do novo
  sucessor do destinatário.

\item \textbf{STORE} Esta mensagem é utilizada para armazenar um valor
  sob uma determinada chave. Como primeiro argumento ela recebe a
  chave do valor a ser armazenada no formato de um inteiro decimal sem
  sinal. Em seguida, ela recebe um número \texttt{n} que indica o
  número de bytes do objeto a ser transferido. Em seguida, são
  passados \texttt{n} inteiros com valores entre 0 a 255 (inclusive)
  representando cada um dos bytes do objeto a ser armazenado. A
  mensagem, como as demais, acaba no final da linha que é demarcado
  por um \texttt{\textbackslash{}n}.

  Note que esta maneira de representar os dados binários a serem
  transferidos não é nada eficiente. Para cada byte do dado sendo
  enviado serão transferidos até 3 bytes em sua representação como
  decimal em texto puro. Existem outras maneiras bem mais eficientes,
  utilizando apenas texto puro, que poderiam ser utilizadas para
  transferir uma sequência de bytes (por exemplo,
  Base64\footnote{\url{https://en.wikipedia.org/wiki/Base64}}). Caso
  deseje, fique à vontade para alterar o protocolo para
  utilizá-las. Isto será bem recompensado durante a correção.

\item \textbf{RETRIEVE} Esta mensagem é usada para recuperar valores
  armazenados na DHT. Ela recebe como argumentos a chave sendo buscada
  representada por um inteiro sem sinal (hash da chave do objeto sendo
  buscado), o identificador do nó que está fazendo a busca, seu
  endereço IP e porta.

\item \textbf{OK} Esta mensagem é enviada em resposta a uma mensagem
  \texttt{RETRIEVE} pelo nó responsável pela chave sendo buscada
  quando ela está presente na DHT. Ela recebe como argumentos a chave
  do objeto, \texttt{n} que é o seu tamanho em bytes e uma
  sequência de \texttt{n} inteiros entre 0 e 255 (inclusive), no mesmo
  formato utilizado pela mensagem \texttt{STORE}. A mensagem como as
  demais, acaba no final da linha que é demarcado por um
  \texttt{\textbackslash{}n}.

\item \textbf{NOT\_FOUND} Esta mensagem é enviada em resposta a uma
  mensagem \texttt{RETRIEVE} pelo nó responsável pela chave buscada
  quando não há nenhum dado armazenado com a chave procurada. Esta
  mensagem não tem nenhum argumento.

\item \textbf{TRANSFER} Esta mensagem transfere a responsabilidade
  pelo armazenamento de um par chave valor entre os nós da DHT. Uma
  mensagem \texttt{TRANSFER} recebe como primeiro argumento a chave do
  valor armazenado no formato já descrito acima, seguido do próprio
  valor no mesmo formato utilizado pelas mensagens \texttt{STORE} e
  \texttt{OK}. Tantas mensagens \texttt{TRANSFER} quanto forem
  necessárias podem ser enviadas em sequência durante o processo de
  ingresso ou partida de um determinado nó na DHT.

\end{itemize}

As mensagens \texttt{JOIN}, \texttt{STORE} e \texttt{RETRIEVE}
\textbf{devem ser roteadas através do anel} até alcançarem o nó onde
serão efetivamente processadas.

As mensagens \texttt{JOIN\_OK}, \texttt{OK}, \texttt{LEAVE},
\texttt{NEW\_NODE}, \texttt{NODE\_GONE}, \texttt{TRANSFER} e
\texttt{NOT\_FOUND} \textbf{devem ser enviadas diretamente ao
  destinatário, sem serem roteadas pelo anel}.

Atente-se para o fato de que, (i) excetuado-se a função
\texttt{join(KnownHostList)} da API, o protocolo não tem nenhum
tratamento especial para o caso de falhas de nós. Caso um nó saia da
rede sem que o processo iniciado com o envio da mensagem
\texttt{LEAVE} seja terminado a rede ficará em um estado
inconsistente. E (ii) também poderia ocorrer de dois ou mais nós
tentarem se conectar simultaneamente à DHT. Neste caso, se pelo menos
dois dos nós ingressantes compartilharem o mesmo sucessor o protocolo
de conexão descrito acima falharia. Para este projeto específico
\textbf{não será necessário} incluir o código para o tratamento destes
casos, contudo é importante estar ciente que uma DHT real precisaria
necessariamente lidar com estes e outros casos de concorrência e
falhas.

\subsection{Roteamento de mensagens e manutenção de conexões}

O roteamento de mensagens \texttt{JOIN}, \texttt{STORE} e
\texttt{RETRIEVE} no anel da DHT deve ser feito de maneira
recursiva\footnote{[ST] Capítulo 5, Seção 5.3.}. Isto significa que
quando um nó recebe uma mensagem cujo destinatário é outro nó, ele
deve encaminhá-la para o seu sucessor que repetirá o processo caso
necessário. Quando a mensagem finalmente alcançar o seu destinatário,
ele a processa e a responde enviando a resposta ao seu predecessor
até que ela alcance o nó que originou a comunicação, seguindo a
sequência reversa dos nós envolvidos no envio da mensagem original.

Já as demais mensagens (\texttt{JOIN\_OK}, \texttt{OK}, \texttt{LEAVE},
\texttt{NEW\_NODE}, \texttt{NODE\_GONE}, \texttt{TRANSFER} e
\texttt{NOT\_FOUND}) podem ser enviadas diretamente ao destinatário via
TCP/IP (ou RMI caso este seja o método de comunicação escolhido).

Caso você tenha escolhido TCP/IP para o envio das suas mensagens, note
que você não precisa necessariamente manter uma conexão aberta entre
os nós e seus sucessores e predecessores o tempo todo: os endereços IP
e portas são conhecidos e você pode sempre abrir uma nova conexão
quando necessário. Isso vai simplificar muito a sua implementação.

\subsection{Funções de hash}

Para obter os identificadores dos nós e dos objetos você pode utilizar
uma função de hash pronta dentre as disponíveis na linguagem de
programação da sua escolha. Favoreça funções de hash
criptograficamente seguras (disponíveis na maior parte das linguagens
de programação) pois elas garantem que os hashes serão uniformemente
distribuídos. Como sugestão para calcular o identificador de um nó,
utilize o endereço IP concatenado com a porta que ele está
utilizando. Durante os testes você pode utilizar funções mais simples
de hash (ou até mesmo valores baixos escolhidos manualmente) para
poder avaliar o funcionamento da sua implementação.


\subsection{Dicas para testar o seu código}

Quando estiver testando o seu código utilize pelo menos 4 nós. Não há
problema algum caso todos os nós estejam no mesmo computador, apenas
preste atenção para que cada um deles esteja usando um número de porta
diferente. Também deixe de uma maneira relativamente fácil no seu
código uma funcionalidade para especificar o identificador do nó e de
um dado a ser armazenado na DHT manualmente. Como os identificadores
gerados pelas funções de hash são longos isso pode facilitar a
depuração de eventuais bugs.

\section{Milestones}

A seguir descrevemos uma série de \emph{milestones} cuja ordem
sugerimos que seja seguida pois ela facilitará o desenvolvimento
incremental do projeto.

\paragraph{Milestone 1}
Escreva uma aplicação simples que usa a API definida na
Seção~\ref{sec:dht}. A aplicação deverá prover uma interface (texto ou
GUI). A aplicação deverá utilizar todas as operações descritas na API
da DHT.

\paragraph{Milestone 2}
Implemente o anel da DHT de modo que ele possa ser criado e
mantido. Mais especificamente, este \emph{milestone} requer que seu
código seja capaz de responder corretamente as mensagens
\texttt{JOIN}, \texttt{JOIN\_OK}, \texttt{NEW\_NODE}, \texttt{LEAVE} e
\texttt{NODE\_GONE}. Com o tratamento dessas mensagens implementado, as
operações \texttt{join()} e \texttt{leave()} da DHT já devem funcionar
corretamente. Como neste ponto não haverá nenhuma chave armazenada nos
nós da DHT, não haverá o envio de mensagens \texttt{TRANSFER} entre os
nós.

\paragraph{Milestone 3}
Neste \emph{milestone} você deverá escrever o código para dar suporte
ao armazenamento e recuperação de objetos da DHT. Para isto o resto
das mensagens definidas pelo protocolo deverão ser implementadas.


\section{Instruções}

As instruções abaixo devem ser seguidas à risca:

\begin{enumerate}
\item[GRUPOS] O projeto deve ser feito em \emph{equipes de até 4
    pessoas}. A lista com os integrantes de cada grupo deve ser
  enviada ao professor da sua turma de prática pelo TIDIA em tarefa
  criada especificamente com este propósito até \textbf{no máximo dia
    30/07/2018}.
  \begin{itemize}
  \item Grupos que não informarem os seus integrantes no prazo acima
    terão \textbf{1 ponto deduzido na nota final do projeto}.
  \end{itemize}

\item[DÚVIDAS] Questões e discussões relativas ao enunciado do
  projeto ou sua implementação devem ser discutidas no fórum do
  TIDIA. Todos são \textbf{fortemente encorajados} a participar das
  discussões e ajudar seus colegas.

\item[RELATÓRIO] Entregue junto com o projeto um relatório detalhado descrevendo
  a solução implementada e alguns exemplos de uso da interface do
  projeto. Por exemplo, se você implementar um \textit{prompt} de
  comando, cada exemplo consistiria de um sessão com os comandos
  necessários e suas respectivas saídas. Os cenários serão testados
  durante a correção, portanto assegure-se que eles estão completos e
  autoexplicativos. Seu relatório deve ser enviado no formato PDF e
  deve ter \textbf{no máximo 4 páginas}.
  \begin{itemize}
  \item Relatórios que não descrevem detalhadamente os passos
    necessários para compilar e executar o projeto poderão sofrer uma
    redução significativa na nota final.
  \item Projetos com erros de compilação receberão nota 0
  \end{itemize}

\item[ENTREGA] A entrega deve ser feita no prazo indicado no
  início deste documento em um único arquivo contendo todo o código
  fonte, relatório e arquivos relevantes via TIDIA em uma atividade
  que será criada exclusivamente para este fim. Lembre-se que seu
  código também será avaliado pela organização, clareza e comentários.

  \begin{itemize}
  \item Entregas de projetos com atraso serão aceitas. Contudo a cada
    dia de atraso a sua nota máxima será atualizada conforme abaixo:
    \begin{table}[h]
      \centering
      \begin{tabular}{|c|c|}
        \hline
        \textbf{Dias em Atraso} & \textbf{Nota Máxima} \\ \hline
        0                       & 10                   \\ \hline
        1                       & 8                    \\ \hline
        2                       & 7                    \\ \hline
        3                       & 6                    \\ \hline
        4                       & 5                    \\ \hline
        5 ou mais               & 0                    \\ \hline
      \end{tabular}
    \end{table}
  \end{itemize}


\end{enumerate}
\
\section{Datas Importantes}
\begin{itemize}
\item \textbf{30/07/2018} - Informar ao professor da prática (via TIDIA) os
  RAs e Nomes dos participantes de cada grupo.
\item \textbf{27/08/2018} - Entrega via TIDIA do projeto final.
\end{itemize}

\vfill \emph{Esse projeto é uma adaptação daquele proposto pelo
  Prof. Jussi Kangasharju da Universidade de
  Helsinki\footnote{\url{https://www.cs.helsinki.fi/en/courses/582665/2012/K/K/1}}
  para o curso de Sistemas Distribuídos.}



\end{document}

%%% Local Variables:
%%% mode: latex
%%% TeX-engine: xetex
%%% TeX-master: t
%%% TeX-command-extra-options: "-shell-escape"
%%% ispell-local-dictionary: "brasileiro"
%%% End:

% LocalWords:  prompt hash DHT Chord API m-bits finger table TCP JRMI
% LocalWords:  sockets retrieve Milestones Milestone GUI milestones
% LocalWords:  RMI bugs milestone hashes
