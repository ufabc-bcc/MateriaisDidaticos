\documentclass[oneside,12pt,a4paper]{memoir}

\usepackage{polyglossia}
\setdefaultlanguage{brazil}
%\disablehyphenation

\usepackage{fontspec}
\setmainfont[Ligatures=TeX]{Linux Libertine O} % Calibri?
\setsansfont[Ligatures=TeX]{Linux Biolinum O}
\setmonofont[Ligatures=TeX]{Inconsolata}

\usepackage[xetex]{hyperref}

\usepackage{minted}

\title{Projeto 1\\ Java Remote Method Invocation (RMI)}
\author{Profs. Emilio Francesquini e Fernando Teubl Ferreira\\
  \texttt{\{e.francesquini,fernando.teubl\}@ufabc.edu.br} \\
  \mbox{}\\
  Centro de Matemática, Computação e Cognição\\
  Universidade Federal do ABC}

\date{Entrega: \textbf{16 de julho de 2018}}

\begin{document}

\maketitle

\section*{Descrição Geral}

Vamos construir uma ``versão enxuta'' de um sistema de informações
sobre ``peças'' ou ``componentes'' (\textit{parts}) usando Remote
Method Invocation (RMI) de Java. O sistema será distribuído por
múltiplos servidores, cada qual implementando um repositório de
informações sobre peças. Cada peça será representada por um objeto
cuja interface é \texttt{Part}. Cada servidor implementará um objeto
\texttt{PartRepository}, que é essencialmente uma coleção de
\texttt{Part}s.

As interfaces \texttt{Part} e \texttt{PartRepository} devem ser
definidas por vocês e são parte da implementação do projeto.

\paragraph{\texttt{Part}}

Cada objeto \texttt{Part} encapsula as seguintes informações:

\begin{itemize}
\item o código da peça, um identificador automaticamente gerado pelo sistema na ocasião da inserção das informações sobre a peça;
\item o nome da peça;
\item a descrição da peça;
\item a lista de subcomponentes da peça.
\end{itemize}

Uma peça pode ser uma agregação de subcomponentes ou pode ser uma peça
primitiva (não composta por subpeças). Sua lista de subcomponentes
contém pares (\texttt{subPart}, \texttt{quant}), onde \texttt{subPart}
referencia um subcomponente da peça, e \texttt{quant} indica quantas
unidades do subcomponente aparecem na peça. Uma peça primitiva tem sua
lista de componentes vazia.

Os subcomponentes de um objeto \texttt{Part} agregado são também
objetos \texttt{Part}. Esses objetos não são necessariamente
implementados pelo mesmo servidor que implementa a peça agregada. Eles
podem estar distribuídos por múltiplos servidores.

\paragraph{\texttt{PartRepository}}

Os objetos \texttt{PartRepository} devem implementar repositórios de
peças, isso é, servidores para o acesso à conjuntos de peças. Em
particular, você deve ser capaz de inserir uma nova \texttt{Part} ao
repositório, recuperar uma \texttt{Part} pelo seu código e obter uma
lista de todas as \texttt{Part}s que estão armazenadas em um dado
repositório.

Neste projeto, apenas os servidores implementados por
\texttt{PartRepository} devem ser registrados e recuperados do serviço
de nomes do Java RMI.

\section*{Projeto}

Cada equipe de projeto escreverá um ``programa servidor'' e um
``programa cliente''.

\subsection*{O Servidor}

O programa servidor implementará as interfaces \texttt{PartRepository}
e \texttt{Part}. Escreva-o tendo em mente que poderão ocorrer várias
execuções simultâneas do programa servidor: cada ``processo servidor''
(uma execução do programa servidor) implementará um objeto
\texttt{PartRepository}, mais a correspondente coleção de objetos
\texttt{Part}. Isto significa que o programa servidor deve receber
como argumentos, na linha de comando, certos parametros que devem
variar de um processo servidor para outro (o nome do servidor, por
exemplo).

\subsection*{O Cliente}

O programa cliente será usado para exercitar o sistema. Ele deve
permitir que o usuário:

\begin{itemize}
\item estabeleça uma conexão com um (processo) servidor;
\item interaja com o repositório implementado pelo servidor:
  \begin{itemize}
  \item examinando o nome do repositório e o numero de peças nele contidas,
  \item listando as peças no repositório,
  \item buscando uma peça (por código de peça) no repositório,
  \item adicionando ao repositório novas peças (primitivas ou agregadas);
  \end{itemize}
\item tendo uma referência a uma peça, referência essa previamente obtida como resultado de uma busca num repositório, interaja com a peça:
  \begin{itemize}
  \item examinando o nome e a descrição da peça,
  \item obtendo o (nome do) repositório que a contém,
  \item verificando se a peça é primitiva ou agregada,
  \item obtendo o número de subcomponentes diretos e primitivos da peça,
  \item listando suas subpeças.
  \end{itemize}
\end{itemize}

Fique à vontade para definir como seu programa cliente vai fazer a
interface com os usuários. O único requisito é que a interface com o
usuário permita que o sistema seja exercitado da forma descrita
acima. Em particular, o programa cliente deve possibilitar que um
usuário crie (de modo razoavelmente conveniente) peças agregadas cujas
subpeças estejam distribuídas por vários repositórios. Provavelmente o
mais fácil é escrever um cliente com uma interface tipo linha de
comando. Uma possibilidade é um cliente ``linha de comando'' que
mantenha três variáveis:

\begin{itemize}
\item o ``repositório atual'', uma referência ao repositório com o
  qual toda interação ocorre;
\item a ``peça atual'', uma referência à peça com a qual toda
  interaçao ocorre;
\item a ``lista de subpeças atual'', usada exclusivamente quando
  uma nova peça é adicionada ao repositório atual.
\end{itemize}

Tal cliente apresentaria um \textit{prompt} e ficaria esperando
comandos do usuário. Ele aceitaria comandos como:

\begin{description}
\item[bind] Faz o cliente se conectar a outro servidor e muda o
  repositório atual. Este comando recebe o nome de um repositório e
  obtém do serviço de nomes uma referência para esse repositório, que
  passa a ser o repositório atual.
\item[listp] Lista as peças do repositório atual.
\item[getp] Busca uma peça por código. A busca é efetuada no
  repositório atual. Se encontrada, a peça passa a ser a nova peça
  atual.
\item[showp] Mostra atributos da peça atual.
\item[clearlist] Esvazia a lista de subpeças atual.
\item[addsubpart] Adiciona à lista de subpeças atual n unidades da
  peça atual.
\item[addp] Adiciona uma peça ao repositório atual. A lista de
  subpeças atual é usada como lista de subcomponentes diretos
  da nova peça. (É só para isto que existe a lista de subpeças
  atual.)
\item[quit] Encerra a execução do cliente.
\end{description}

A lista acima tem a finalidade de ilustrar como um cliente ``linha de
comando'' poderia funcionar. Tome-a como uma sugestão (incompleta, por
sinal), que pode ser seguida ou não. Se você tiver gás para escrever
um cliente com uma interface com o usuário mais elaborada e amigável
(GUI), vá em frente!

\section*{Instruções}

As instruções abaixo devem ser seguidas à risca:

\begin{enumerate}
\item O projeto deve ser feito em \emph{equipes de 2 ou 3 pessoas}. Os
  grupos devem ser informados ao professor da sua turma de prática por
  e-mail até \textbf{no máximo dia 20/06/2018}.

\item Dúvidas em relação ao projeto devem ser discutidas no fórum do
  TIDIA. Todos são \textbf{fortemente encorajados} a participar das
  discussões e ajudar seus colegas.

\item Entregue junto com o projeto um relatório detalhado descrevendo a
  solução implementada e alguns exemplos de uso da interface do
  projeto. Por exemplo, se você implementar um \textit{prompt} de comando,
  cada exemplo consistiria de um sessão com os comandos necessários e
  suas respectivas saídas. Os cenários serão testados durante a correção,
  portanto assegure-se que eles estão completos e autoexplicativos.

\item A entrega deve ser feita no prazo indicado no início deste
  documento em um único arquivo contendo todo o código, relatório e
  arquivos relevantes via TIDIA em uma atividade que será criada
  exclusivamente para este fim.

\item Cuidado ao seguir tutoriais desatualizados na Internet. O
  arcabouço que implementa o Java RMI sofreu várias modificações ao
  longo dos anos. Sua capacidade de encontrar informações confiáveis e
  atualizadas será levada em consideração na atribuição da nota.
\end{enumerate}
\
\section*{Datas Importantes}
\begin{itemize}
\item \textbf{20/06/2018} - Informar ao professor da prática (via e-mail) os
  RAs e Nomes dos participantes de cada grupo.
\item \textbf{16/07/2018} - Entrega via TIDIA do projeto final.
\end{itemize}

\vfill
\emph{Esse projeto é uma adaptação para Java RMI (originalmente proposta
pelo Prof. Daniel Codeiro da EACH-USP) do
projeto\footnote{\url{http://www.ime.usp.br/~reverbel/SOD-03/proj_corba.html}}
proposto pelo prof. Francisco Reverbel (IME/USP) para seu curso com
CORBA.}

\end{document}

%%% Local Variables:
%%% mode: latex
%%% TeX-engine: xetex
%%% TeX-master: t
%%% TeX-command-extra-options: "-shell-escape"
%%% ispell-local-dictionary: "brasileiro"x
%%% End:
